% ACM Algorithm Template - v4.0
% Yume Maruyama <kirainmoe@gmail.com>

\documentclass[12pt]{report}

\usepackage{fontspec}
\usepackage{graphicx}
\usepackage{geometry}
\usepackage{listings}
\usepackage[dvipsnames, svgnames, x11names]{xcolor}
\usepackage{fontspec}
\usepackage{mathtools}
\usepackage{verbatim} 
\usepackage{amsthm}
\usepackage{fancyhdr}
\usepackage{minted}
\usepackage{verbatim}
\usepackage{pdfpages}
\usepackage{setspace}  %设置行间距
\usepackage[colorlinks=true]{hyperref}
\usepackage[BoldFont,SlantFont,CJKchecksingle]{xeCJK}
\usepackage{tcolorbox}
\usepackage{lastpage} %总页数
\usepackage{fancyhdr} %使用fancyhdr

% 设置字体
\setCJKmainfont{SimSun}
\setmainfont{Times New Roman}
\setsansfont{Fira Code}
% 页边距,行距设置
\geometry{left=1.5cm,right=1.5cm,top=1.5cm,bottom=1.5cm}
\linespread{1.2}

% listings 样式
\lstset{
    basicstyle = \ttfamily,
}

% 代码样式,使用 minted
\setminted[c++]{linenos, frame=lines, framesep=2mm, baselinestretch=1.0, breaklines, mathescape, numbers=left, numbersep=2mm}
\setminted[java]{linenos, frame=leftline, framesep=2mm, baselinestretch=1.0, breaklines, mathescape, bgcolor=LavenderBlush1}
\setminted[python]{linenos, frame=leftline, framesep=2mm, baselinestretch=1.0, breaklines, mathescape, bgcolor=LavenderBlush1}
\setminted[batch]{linenos, frame=leftline, framesep=2mm, baselinestretch=1.0, breaklines, mathescape, bgcolor=LavenderBlush1}
\setminted[bash]{linenos, frame=leftline, framesep=2mm, baselinestretch=1.0, breaklines, mathescape, bgcolor=LavenderBlush1}
\setminted[json]{linenos, frame=leftline, framesep=2mm, baselinestretch=1.0, breaklines, mathescape, bgcolor=LavenderBlush1}
\usemintedstyle{vs}
\definecolor{primary-blue}{RGB}{9,109,217}
\definecolor{cyan}{RGB}{19,194,194}

% 标题 & 作者信息
\title{\Huge \textbf{ICPC Templates} \linebreak \linebreak \large \textcolor{primary-blue}{\textbf{我们需要更深入浅出一些}}}
\author{Mine\_King, Tx\_Lcy, 369pai \\ \small 洛谷科技}

% 设置页眉风格
\pagestyle{fancy}                         %设置页眉页脚
\lhead{}                                  %页眉左侧显示页数                  
\chead{}                                  %页眉中
\rhead{\small\leftmark}                   %章节信息                       
\cfoot{\thepage}                          %当前页,记得调用前文提到的宏包 
\rfoot{}%                                 %页脚左               
\lfoot{}%					          %页脚右
\renewcommand{\headrulewidth}{0mm}        %页眉线宽,设为0可以去页眉线
\renewcommand{\footrulewidth}{0mm}        %页脚线宽,设为0可以去页脚线

\begin{document}
% 用\setlength设置页眉页脚边距
%\setlength{\voffset}{-10mm}                        
%\setlength{\topmargin}{0mm}
\setlength{\headheight}{5mm}
\setlength{\headsep}{5mm}
%\setlength{\footskip}{10mm}
%\setlength{\baselineskip}{20pt}  %设置行间距

% 标题页面
\begin{titlepage}
    \begin{figure}
        \centering
        \includegraphics[height=8cm]{images/logo.jpg}
        % \par\includegraphics[height=3cm]{images/acm.jpg}
    \end{figure}
    
    \maketitle
\end{titlepage}

%\clearpage
%\phantom{s}
%\setcounter{page}{0}
%\thispagestyle{empty}
%\clearpage

% 目录
\tableofcontents       % 生成目录
\thispagestyle{empty}  % 目录页不显示页码

\newpage
\setcounter{page}{1}   % 从下面开始编页码

% 正文内容

%% Chapter ? 字符串
\chapter{字符串}

\section{最小表示法}

\begin{minted}{c++}
int i = 0, j = 1, k = 0;
while (k < n && i < n && j < n)
    if (a[(i + k) % n] == a[(j + k) % n]) k++;
    else {
        if (a[(i + k) % n] > a[(j + k) % n]) i = i + k + 1;
        else j = j + k + 1;
        if (i == j) i++;
        k = 0;
    }
ans = min(i, j);
\end{minted}

\section{Border 理论}

\subsection{关键结论}

\begin{tcolorbox}
\textbf{定理 1}:对于一个字符串 $s$,若用 $t$ 表示其最长的 Border,则有 $\mathcal{B}(s) = \mathcal{B}(t) \cup \{t\}$。
\end{tcolorbox}

\begin{tcolorbox}
\textbf{定理 2}:一个字符串的 Border 与 Period 一一对应。具体地,$\mathrm{pre}(s, i) \in \mathcal{B}(s) \iff |s| - i \in \mathcal{P}(s)$。
\end{tcolorbox}

\begin{tcolorbox}
\textbf{弱周期引理}:

$$
\forall p, q\in\mathcal{P}(s), p + q \le |s| \implies \gcd(p, q)\in\mathcal{P}(s)
$$
\end{tcolorbox}

\begin{tcolorbox}
\textbf{定理 3}:若字符串 $t$ 是字符串 $s$ 的前缀,且 $a \in \mathcal{P}(s), b \in \mathcal{P}(t), b \mid a, |t| \ge a$,且 $b$ 是 $t$ 的整周期,则有 $b \in \mathcal{P}(s)$。
\end{tcolorbox}

\begin{tcolorbox}
\textbf{周期引理}:

$$
\forall p, q\in\mathcal{P}(s), p + q - \gcd(p, q) \le |s| \implies \gcd(p, q)\in\mathcal{P}(s)
$$
\end{tcolorbox}

\begin{tcolorbox}
\textbf{定理 4}:对于文本串 $s$ 和模式串 $t$,若 $|t| \ge \frac{|s|}{2}$,且 $t$ 在 $s$ 中至少成功匹配了 $3$ 次,则每次匹配的位置形成一个等差数列,且公差为 $t$ 的最小周期。
\end{tcolorbox}

\begin{tcolorbox}
\textbf{定理 5}:一个字符串 $s$ 的所有长度不小于 $\frac{|s|}{2}$ 的 Border 的长度构成一个等差数列。
\end{tcolorbox}

\begin{tcolorbox}
\textbf{定理 6}:一个字符串的所有 Border 的长度排序后可以划分成 $\lceil\log_2|s|\rceil$ 个连续段,使得每段都是一个等差数列。
\end{tcolorbox}

\begin{tcolorbox}
\textbf{定理 7}:回文串的回文前/后缀即为该串的 Border。
\end{tcolorbox}

\begin{tcolorbox}
\textbf{定理 8}:若回文串 $s$ 有周期 $p$,则可以把 $\mathrm{pre}(s, p)$ 划分成长度为 $|s| \bmod p$ 的前缀和长度为 $p - |s| \bmod p$ 的后缀,使得它们都是回文串。
\end{tcolorbox}

\begin{tcolorbox}
\textbf{定理 9}:若 $t$ 是回文串 $s$ 的最长 Border 且 $|t| \ge \frac{|s|}{2}$,则 $t$ 在 $s$ 中只能匹配 $2$ 次。
\end{tcolorbox}

\begin{tcolorbox}
\textbf{定理 10}:对于任意一个字符串以及 $u, v\in\mathrm{Ssuf}(s), |u| < |v|$,一定有 $u$ 是 $v$ 的 Border。
\end{tcolorbox}

\begin{tcolorbox}
\textbf{定理 11}:对于任意一个字符串 $s$ 以及 $u, v\in\mathrm{Ssuf}(s), |u| < |v|$,一定有 $2|u| \le |v|$。
\end{tcolorbox}

\begin{tcolorbox}
\textbf{定理 12}:$|\mathrm{Ssuf}(s)| \le \log_2|s|$。
\end{tcolorbox}

\subsection{KMP}

\begin{minted}{c++}
// n is |s|, m is |t|.
for (int i = 1, j = 0; i <= n; i++) {
    while (j && t[j + 1] != s[i]) j = pi[j];
    if (t[j + 1] == s[i])
        if (++j == m) {
            // ...
            j = nxt[j];
        }
}
\end{minted}

\section{回文自动机 PAM}

\begin{minted}{c++}
// == Main ==
struct PAM {
    int tot, delta[500005][26], len[500005], fail[500005], ans[500005];
    string s;
    int lst;

    PAM() {tot = 1; len[0] = 0; len[1] = -1; fail[0] = fail[1] = 1;}
    int getfail(int now, int i) {
        while (s[i - len[now] - 1] != s[i]) now = fail[now];
        return now;
    }
    void insert(int i) {
        int now = getfail(lst, i);
        if (!delta[now][s[i] - 'a']) {
            len[++tot] = len[now] + 2;
            fail[tot] = delta[getfail(fail[now], i)][s[i] - 'a'];
            delta[now][s[i] - 'a'] = tot;
            ans[tot] = ans[fail[tot]] + 1;
        }
        lst = delta[now][s[i] - 'a'];
        return;
    }
} p;
\end{minted}

\section{后缀自动机}

\subsection{普通 SAM}

\begin{minted}{c++}
// == Main ==
struct SAM {
    int tot, lst;
    int len[2000005], siz[2000005], link[2000005];
    int delta[2000005][26];

    SAM() {link[0] = -1;}
    void insert(char ch) {
        int c = ch - 'a', now = ++tot;
        len[now] = len[lst] + 1;
        siz[now] = 1;
        for (int p = lst; p != -1; p = link[p])
            if (!delta[p][c]) delta[p][c] = tot;
            else if (len[delta[p][c]] == len[p] + 1) {link[now] = delta[p][c]; break;}
            else {
                int q = delta[p][c], v = ++tot;
                len[v] = len[p] + 1;
                memcpy(delta[v], delta[q], sizeof(delta[v]));
                link[v] = link[q], link[q] = v, link[now] = v;
                for (int i = p; delta[i][c] == q; i = link[i]) delta[i][c] = v;
                break;
            }
        lst = now;
        return ;
    }
} sam;
\end{minted}

\subsection{广义 SAM}

注意自动机空间要开 Trie 的两倍。

\begin{minted}{c++}
// == Main ==
struct GSAM {
    int tot;
    int delta[2000005][26], link[2000005], len[2000005];
    struct Trie {
        int tot, trie[1000005][26], st[1000005];

        void insert(string s) {
            int now = 0;
            for (char c : s) {
                int id = c - 'a';
                if (!trie[now][id]) trie[now][id] = ++tot;
                now = trie[now][id];
            }
            return;
        }
    } tr;

    GSAM() {link[0] = -1;}
    int insert(int c, int lst) {
        int now = ++tot;
        len[now] = len[lst] + 1;
        for (int p = lst; p != -1; p = link[p])
            if (!delta[p][c]) delta[p][c] = now;
            else if (len[delta[p][c]] == len[p] + 1) {link[now] = delta[p][c]; break;}
            else {
                int q = delta[p][c], v = ++tot;
                len[v] = len[p] + 1;
                memcpy(delta[v], delta[q], sizeof(delta[v]));
                link[v] = link[q], link[q] = v, link[now] = v;
                for (int i = p; i != -1 && delta[i][c] == q; i = link[i]) delta[i][c] = v;
                break;
            }
        return now;
    }
    void build() {
        queue<int> q;
        tr.st[0] = 0;
        q.push(0);
        while (!q.empty()) {
            int now = q.front();
            q.pop();
            for (int i = 0; i < 26; i++)
                if (tr.trie[now][i])
                    tr.st[tr.trie[now][i]] = insert(i, tr.st[now]), q.push(tr.trie[now][i]);
        }
        return;
    }
} gsam;
\end{minted}

\section{后缀排序}

\begin{minted}{c++}
// == Preparations ==
int sa[2000005], rk[2000005], b[1000005], cp[2000005];
// == Main ==
for (int i = 1; i <= n; i++) b[rk[i] = s[i]]++;
for (int i = 1; i < 128; i++) b[i] += b[i - 1];
for (int i = n; i >= 1; i--) sa[b[rk[i]]--] = i;
memcpy(cp, rk, sizeof(cp));
for (int i = 1, j = 0; i <= n; i++)
    if (cp[sa[i]] == cp[sa[i - 1]]) rk[sa[i]] = j;
    else rk[sa[i]] = ++j;
for (int w = 1; w < n; w <<= 1) {
    memcpy(cp, sa, sizeof(cp));
    memset(b, 0, sizeof(b));
    for (int i = 1; i <= n; i++) b[rk[cp[i] + w]]++;
    for (int i = 1; i <= n; i++) b[i] += b[i - 1];
    for (int i = n; i >= 1; i--) sa[b[rk[cp[i] + w]]--] = cp[i];
    memcpy(cp, sa, sizeof(cp));
    memset(b, 0, sizeof(b));
    for (int i = 1; i <= n; i++) b[rk[cp[i]]]++;
    for (int i = 1; i <= n; i++) b[i] += b[i - 1];
    for (int i = n; i >= 1; i--) sa[b[rk[cp[i]]]--] = cp[i];
    memcpy(cp, rk, sizeof(cp));
    for (int i = 1, j = 0; i <= n; i++)
        if (cp[sa[i]] == cp[sa[i - 1]] && cp[sa[i] + w] == cp[sa[i - 1] + w]) rk[sa[i]] = j;
        else rk[sa[i]] = ++j;
}
\end{minted}

求 height:

\begin{minted}{c++}
for (int i = 1, k = 0; i <= n; i++) {
    if (k) k--;
    while (i + k <= n && sa[rk[i] - 1] + k <= n && s[i + k] == s[sa[rk[i] - 1] + k]) k++;
    height[rk[i]] = k;
}
\end{minted}

$$
LCP(sa_i, sa_j) = \min\limits_{k = i + 1}^j \{height_k\}
$$
%
%% Chapter ? 数论
\chapter{数论}

\section{Miller Rabin 和 Pollard Rho}

\subsection{Miller Rabin}

\begin{minted}{c++}
// == Preparations ==
const int prime[] = {2, 3, 5, 7, 9, 11, 13, 17, 19, 23, 29, 31, 37};

long long power(long long a, long long b, long long mod) {
    long long ans = 1;
    while (b) {
        if (b & 1) ans = (__int128)ans * a % mod;
        a = (__int128)a * a % mod;
        b >>= 1;
    }
    return ans % mod;
}
// == Main ==
inline int Miller_Rabin(long long n) {
    if (n == 1) return 0;
    if (n == 2) return 1;
    if (n % 2 == 0) return 0;
    long long u = n - 1, t = 0;
    while (u % 2 == 0) u /= 2, t++;
    for (int i = 0; i < 12; i++) {
        if (prime[i] % n == 0) continue;
        long long x = power(prime[i] % n, u, n);
        if (x == 1) continue;
        int flag = 0;
        for (int j = 1; j <= t; j++) {
            if (x == n - 1) {flag = 1; break;}
            x = (__int128)x * x % n;
        }
        if (!flag) return 0;
    }
    return 1;
}
\end{minted}

\subsection{Pollard Rho}

\begin{minted}{c++}
// == Preparations ==
#include <chrono>
#include <random>

mt19937_64 gen(chrono::system_clock::now().time_since_epoch().count());
/*
Miller Rabin
*/
// == Main ==
long long Pollard_Rho(long long n) {
    long long s = 0, t = 0, c = gen() % (n - 1) + 1;
    for (int goal = 1; ; goal <<= 1, s = t) {
        long long val = 1;
        for (int step = 1; step <= goal; step++) {
            t = ((__int128)t * t + c) % n;
            val = (__int128)val * abs(t - s) % n;
            if (!val) return n;
            if (step % 127 == 0) {
                long long d = __gcd(val, n);
                if (d > 1) return d;
            }
        }
        long long d = __gcd(val, n);
        if (d > 1) return d;
    }
}
void factor(long long n) {
    if (n < ans) return ;
    if (n == 1 || Miller_Rabin(n)) {
        // n = 1 或 n 是一个质因子。
        // ...
    }
    long long p;
    do p = Pollard_Rho(n);
    while (p == n);
    while (n % p == 0) n /= p;
    factor(n), factor(p);
    return ;
}
\end{minted}

\section{常用数论算法}

\subsection{exgcd}

求出来的解满足 $|x| \le b, |y| \le a$。

\begin{minted}{c++}
// == Main ==
int exgcd(int a, int b, int &x, int &y) {
    if (b == 0) {x = 1; y = 0; return a;}
    int m = exgcd(b, a % b, y, x);
    y -= a / b * x;
    return m;
}
\end{minted}

\subsection{CRT}

\textbf{没用。}

\begin{minted}{c++}
// == Preparations ==
#include <vector>
// == Main ==
int CRT(vector<pair<int, int>> &a) {
    int M = 1;
    for (auto i : a) M *= i.second;
    int res = 0;
    for (auto i : a) {
        int t = inv(M / i.second, i.second);
        res = (res + (long long)i.first * (M / i.second) % M * t % M) % M;
    }
    return res;
}
\end{minted}

\subsection{exCRT}

需保证 $\mathrm{lcm}$ 在 \lstinline|long long| 范围内。

\begin{minted}{c++}
// == Preparations ==
long long exgcd(long long a, long long b, long long &x, long long &y);
// == Main ==
long long exCRT(vector<pair<long long, long long>> vec) {
    long long ans = vec[0].first, mod = vec[0].second;
    for (int i = 1; i < (int)vec.size(); i++) {
        long long a = mod, b = vec[i].second, c = vec[i].first - ans % b;
        long long x, y;
        long long g = exgcd(a, b, x, y);
        if (c % g != 0) return -1;
        b /= g;
        x = (__int128)x * (c / g) % b;
        ans += x * mod;
        mod *= b;
        ans = (ans % mod + mod) % mod;
    }
    return ans;
}
\end{minted}

\subsection{exLucas}

\begin{minted}{c++}
// == Preparations ==
int power(int a, long long b, int mod);
int exgcd(int a, int b, int &x, int &y);
int CRT(vector<pair<int, int>> &a);
// == Main ==
int inv(int n, int p) {
    int x, y;
    exgcd(n, p, x, y);
    return (x % p + p) % p;
}
int fac(long long n, int p, int pk) {
    if (n == 0) return 1;
    int res = 1;
    for (int i = 1; i < pk; i++)
        if (i % p != 0) res = (long long)res * i % pk;
    res = power(res, n / pk, pk);
    for (int i = 1; i <= n % pk; i++)
        if (i % p != 0) res = (long long)res * i % pk;
    return (long long)res * fac(n / p, p, pk) % pk;
}
int C(long long n, long long m, int p, int pk) {
    long long x = n, y = m, z = n - m;
    int res = (long long)fac(x, p, pk) * inv(fac(y, p, pk), pk) % pk * inv(fac(z, p, pk), pk) % pk;
    long long e = 0;
    while (x) e += x / p, x /= p;
    while (y) e -= y / p, y /= p;
    while (z) e -= z / p, z /= p;
    return (long long)res * power(p, e, pk) % pk;
}
int exLucas(long long n, long long m, int p) {
    vector<pair<int, int>> a;
    for (int i = 2; i * i <= p; i++)
        if (p % i == 0) {
            int pk = 1;
            while (p % i == 0) pk *= i, p /= i;
            a.emplace_back(C(n, m, i, pk), pk);
        }
    if (p != 1) a.emplace_back(C(n, m, p, p), p);
    return CRT(a);
}
\end{minted}

\section{万能欧几里得算法}

\begin{tcolorbox}
问题描述:\\
给出一个幺半群 $(S, \cdot)$ 和元素 $u, r \in S$,以及一条直线 $y = \frac{a x + b}{c}$。\\
画出平面中所有坐标为正整数的横线和竖线,维护一个 $f$,初值为单位元 $e$。\\
从原点出发,先向 $y$ 轴正方向走直到到达直线与 $y$ 的交点,然后沿直线走一直走到与 $x = n$ 的交点为止。\\
每当经过一条横线时,执行 $f \gets fu$,经过一条竖线时执行 $f \gets fr$。特别地,在 $y$ 轴上行走时不考虑竖线,同时经过横线和竖线时先执行前者。\\
求最终的 $f$。记为 $\mathrm{euclid}(a, b, c, n, u, r)$。\\
其中 $a, b \ge 0, n, c > 0$。
\end{tcolorbox}

\textbf{做法}:

$$
\mathrm{euclid}(a, b, c, n, u, r) =
\begin{cases}
r ^ n & m = 0 \\
u ^ {\lfloor\frac{b}{c}\rfloor} \cdot \mathrm{euclid}(a \bmod c, b \bmod c, c, n, u, u ^ {\lfloor\frac{a}{c}\rfloor} r) & a \ge c \lor b \ge c \\
r ^ {\lfloor\frac{c - b - 1}{a}\rfloor} u \cdot \mathrm{euclid}(c, (c - b - 1) \bmod a, a, m - 1, r, u) \cdot r ^ {n - \lfloor\frac{c m - b - 1}{a}\rfloor} & \text{otherwise}
\end{cases}
$$

设一次乘法的复杂度为 $O(T)$,则复杂度为 $O(T \log(a + c) \log(a + n + c))$。

\begin{minted}{c++}
// == Preparations ==
struct Node {
    // ...

    Node operator*(const Node &x) const {
        // ...
    }
};
// == Main ==
Node power(Node a, long long b) {
    Node ans = Node(/*幺元*/);
    while (b) {
        if (b & 1) ans = ans * a;
        a = a * a;
        b >>= 1;
    }
    return ans;
}
Node Euclid(int a, int b, int c, long long n, Node r, Node u) {
    long long m = (a * n + b) / c;
    if (!m) return power(r, n);
    if (a >= c || b >= c)
        return power(u, b / c) * Euclid(a % c, b % c, c, n, power(u, a / c) * r, u);
    return power(r, (c - b - 1) / a) * u *
        Euclid(c, (c - b - 1) % a, a, m - 1, u, r) * power(r, n - (c * m - b - 1) / a);
}
\end{minted}

\section{离散对数}

离散对数问题即在模 $p$ 意义下求解 $\log_{a}{b}$。这等价于求解离散对数方程

$$a^x \equiv b\pmod p$$

\subsection{BSGS}

运用了根号分治的思想:设块长为 $M$ 且 $x=AM-B$,则有 $a^{AM} \equiv ba^{B} \pmod p$。

固定模数 $p$ 和底数 $a$ 时,预处理时间为 $\mathcal{O}(\frac{p}{M})$,单次询问 $\mathcal{O}(M)$

如果数据组数为 $T$,当 $M$ 取 $\left \lceil \sqrt{\frac{p}{T}} \right \rceil$ 时取到最优复杂度。

以下代码输入要求 $a \perp p,2\le b,n < p<2^{31}$,求得的是模意义下 $\log_{a}{b}$ 的最小的非负整数解。

\begin{minted}{c++}
// == Preparations ==
int a , p , m , t , sz , phi; 
__gnu_pbds::gp_hash_table<int , int>mp;
int Qpow(int x , int p , int mod);
int Phi(int n);
int Inv(int a , int p){return Qpow(a , phi - 1 , p);}
// == Main ==
void Init()
{
    mp.clear();
    phi = Phi(p); 
    sz = sqrt(phi / m) + 1;
    t = ((ll)phi + sz - 1) / sz;
    int as = Qpow(a , sz , p) , aa = 1;
    for(int i = 1 ; i <= t ; i++)
    {
        aa = (ll)aa * as % p;
        if(!mp[aa])mp[aa] = i;
    }
}
int BSGS(int b)
{
    int r = b , ans = p;
    for(int k = 0 ; k <= sz ; k++)
    {
        if(mp.find(r) != mp.end())
            ans = min((ll)ans , (ll)mp[r] * sz - k);
        r = (ll)r * a % p;
    }
    return ans >= p ? -1 : ans;
}
\end{minted}

\subsection{exBSGS}

将方程化为如下形式,其中 $\frac{p}{D} \perp a$。

$$\frac{a^k}{D}\cdot a^{x-k}\equiv\frac{b}{D} \pmod{\frac{p}{D}}$$

然后使用 BSGS 求解,注意特判 $x \le k$ 的情况。

以下代码输入要求 $1\le a,p,b\le10^9$,求得的是模意义下 $\log_{a}{b}$ 的最小的非负整数解。

\begin{minted}{c++}
// == Preparations ==
int a , p , mod , m , t , sz , phi , k , d , inv; 
__gnu_pbds::gp_hash_table<int , int>mp , low;
int Qpow(int x , int p , int mod);
int Phi(int n);
int Inv(int a , int p);
int BSGS(int b);
// == Main ==
void Init()
{
    mp.clear() , low.clear();
    a %= p , mod = p , d = 1 , k = 0; 
    int ad = 1 , ak = 1; 
    for(int g = __gcd(a , p) ; g != 1 ; g = __gcd(a , p))
    {
        ak = (ll)ak * a % mod , k++;
        if(!low[ak])low[ak] = k;
        d *= g , p /= g , ad = ad * ll(a / g) % p;
    }
    phi = Phi(p) , inv = Inv(ad , p);
    sz = sqrt(phi / m) + 1;
    t = ((ll)phi + sz - 1) / sz;
    int as = Qpow(a , sz , p) , aa = 1;
    for(int i = 1 ; i <= t ; i++)
    {
        aa = (ll)aa * as % p;
        if(!mp[aa])mp[aa] = i;
    }
}
int exBSGS(int b)
{
    int bm = b % mod;
    if(mod == 1 || bm == 1)return 0;    
    if(low.find(bm) != low.end())return low[bm];
    if(b % d)return -1;
    b = (ll)(b / d) * inv % p;
    int ans = BSGS(b);
    return ans == -1 ? -1 : ans + k;
}
\end{minted}

\section{原根}

\subsection{阶}

\begin{tcolorbox}
\textbf{定义}:满足同余式 $a^n \equiv 1 \pmod m$ 的最小正整数 $n$ 存在,这个 $n$ 称作 $a$ 模 $m$ 的阶,记作 $\delta_m(a)$ 或 $\operatorname{ord}_m(a)$。
\end{tcolorbox}

\begin{tcolorbox}
\textbf{性质 1}:$a,a^2,\cdots,a^{\delta_m(a)}$ 模 $m$ 两两不同余。
\end{tcolorbox}

\begin{tcolorbox}
\textbf{性质 2}:若 $a^n \equiv 1 \pmod m$ ,则 $\delta_m(a)\mid n$。
\end{tcolorbox}

\begin{tcolorbox}
\textbf{性质 2 推论}:若 $a^p\equiv a^q\pmod m$ ,则有 $p\equiv q\pmod{\delta_m(a)}$。
\end{tcolorbox}

\begin{tcolorbox}
\textbf{性质 3}:

设 $m\in\mathbf{N}^{*}$ , $a,b\in\mathbf{Z}$ , $(a,m)=(b,m)=1$ ,则

$$\delta_m(ab)=\delta_m(a)\delta_m(b)$$

的充分必要条件是

$$\left(\delta_m(a), \delta_m(b)\right)=1$$
\end{tcolorbox}

\begin{tcolorbox}
\textbf{性质 4}:

设 $k \in \mathbf{N}$ , $m$ 为正整数, $a\in\mathbf{Z}$ , $(a,m)=1$ ,则
$$\delta_m(a^k)=\dfrac{\delta_m(a)}{\left(\delta_m(a),k\right)}$$
\end{tcolorbox}

\subsection{原根}


\begin{tcolorbox}
\textbf{定义}:

设 $m$ 为正整数, $g$ 为整数。 若 $(g,m)=1$ ,且 $\delta_m(g)=\varphi(m)$ ,则称 $g$ 为模 $m$ 的原根。

即 $g$ 满足
$\delta_m(g) = \varphi(m)$。 当 $m$ 是质数时,我们有 $g^i \bmod m,0 < i < m$ 的结果互不相同。
\end{tcolorbox}


\begin{tcolorbox}
\textbf{原根判定定理}:

设 $m \ge 3, (g,m)=1$ ,则 $g$ 是模 $m$ 的原根的充要条件是,对于 $\varphi(m)$ 的每个素因数 $p$ ,都有
$g^{\frac{\varphi(m)}{p}}\not\equiv 1\pmod m$。
\end{tcolorbox}


\begin{tcolorbox}
\textbf{原根个数}:

若一个数 $m$ 有原根,则它原根的个数为 $\varphi(\varphi(m))$。
\end{tcolorbox}


\begin{tcolorbox}
\textbf{原根存在定理}:

一个数 $m$ 存在原根当且仅当 $m=2,4,p^{\alpha},2p^{\alpha}$ ,其中 $p$ 为奇素数, $\alpha\in \mathbf{N}^{*}$。
\end{tcolorbox}

\begin{minted}{c++}
// == Preparations ==
const int N = 1e6 + 5; 
int m , pr[N] , phi[N] , vis[N] , mod; bitset<N>b;
void Sieve(int n = N - 5);
int Qpow(int x , int p , int mod);
// == Main ==
vector<int> PrimitiveRoot(int p , int d)
{
    vector<int>pfac , g; int mod = p;
    if(!vis[p]){return g;} // whether exist primitive root
    for(int i = 1 ; i <= m && pr[i] <= phi[p] ; i++)
        if(phi[p] % pr[i] == 0)pfac.push_back(pr[i]);
    for(int a = 1 ; a < p ; a++) // minimal primitive root is O(n^{0.25+eps})
    {
        if(__gcd(a , p) != 1)continue ;
        bool ok = 1;
        for(int x : pfac)if(Qpow(a , phi[p] / x , mod) == 1){ok = 0; break ;}
        if(!ok)continue ;
        g.push_back(a); break ;
    }
    for(int i = 2 ; i < phi[p] ; i++)
        if(__gcd(i , phi[p]) == 1)g.push_back(Qpow(g[0] , i , mod));
    sort(g.begin() , g.end());
    return g;
}
\end{minted}

\section{剩余}

\begin{tcolorbox}
\textbf{定义}:

令整数 $k\geq 2$ ,整数 $a$ , $m$ 满足 $(a,m)=1$ ,若存在整数 $x$ 使得
\begin{equation}\tag{1}
x^k\equiv a\pmod m 
\end{equation}

则称 $a$ 为模 $m$ 的 $k$ 次剩余,否则称 $a$ 为模 $m$ 的 $k$ 次非剩余。
\end{tcolorbox}

当整数 $k\geq 2$ ,整数 $a$ , $m$ 满足 $(a,m)=1$ ,模 $m$ 有原根 $g$ 时,令 $d=(k,\varphi(m))$ ,则:

\begin{tcolorbox}
\textbf{性质 1}:$a$ 为模 $m$ 的 $k$ 次剩余当且仅当 $d\mid \operatorname{ind}_g a$ ,即:$ a^{\frac{\varphi(m)}{d}}\equiv 1\pmod m $
\end{tcolorbox}

\begin{tcolorbox}
\textbf{证明 1}:令 $x=g^y$ ,则方程 $(1)$ 等价于
$ g^{ky}\equiv g^{\operatorname{ind}_g a}\pmod m $

其等价于
$ ky\equiv \operatorname{ind}_g a\pmod{\varphi(m)} $

由同余的性质,我们知道 $y$ 有整数解当且仅当 $d=(k,\varphi(m))\mid \operatorname{ind}_g a$ ,进而
\begin{align} 
\notag a^{\frac{\varphi(m)}{d}}\equiv 1\pmod m
&\notag \iff g^{\frac{\varphi(m)}{d}\operatorname{ind}_g a}\equiv 1\pmod m\\ 
&\notag \iff \varphi(m)\mid\frac{\varphi(m)}{d}\operatorname{ind}_g a\\ 
&\notag \iff \varphi(m)d\mid \varphi(m)\operatorname{ind}_g a\\ 
&\notag \iff d\mid \operatorname{ind}_g a
\end{align}

\end{tcolorbox}

\begin{tcolorbox}
\textbf{性质 2}:方程 $(1)$ 若有解,则模 $m$ 下恰有 $d$ 个解
\end{tcolorbox}

\begin{tcolorbox}
\textbf{性质 3}:模 $m$ 的 $k$ 次剩余类的个数为 $\dfrac{\varphi(m)}{d}$ , 其有形式
    $$ a\equiv g^{di}\pmod m,\qquad \left(0\leq i<\frac{\varphi(m)}{d}\right) $$
\end{tcolorbox}

\subsection{二次剩余}

下面讨论模数是奇素数时的二次剩余问题:

$$x^2 \equiv n \pmod p$$

参考以上结论不难得出,也就说二次剩余的数量恰为 $\frac{p-1}{2}$ ,其他的非 $0$ 数都是非二次剩余,数量也是 $\frac{p-1}{2}$;判别是否为二次剩余只需检查是否 $n^{\frac{p-1}{2}} \equiv 1 \pmod p$ 即可。

\subsection{Cipolla}

通过随机找到一个 $a$ 满足 $a^2 - n$ 是非二次剩余。

定义 $i^2\equiv a^2 - n$,将所有数表示为 $A+Bi$ 的形式,有 $(a+i)^{p+1}\equiv n$。

那么 $(a+i)^{\frac{p-1}{2}}$ 即是一个解,其相反数是另一个解。

\begin{minted}{c++}
// == Preparations ==
typedef long long ll;
int mod , w2;
mt19937 rnd(114514);
struct Complex
{
    int a , b;
    Complex(ll aa = 0 , ll bb = 0):a(aa % mod) , b(bb % mod){}
};
typedef Complex cp;
cp operator + (cp x , cp y){return cp(x.a + y.a , x.b + y.b);}
cp operator - (cp x , cp y){return cp(x.a - y.a , x.b - y.b);}
cp operator * (cp x , cp y)
{
    return cp((ll)x.a * y.a + (ll)w2 * x.b % mod * y.b ,
              (ll)x.b * y.a + (ll)y.b * x.a);
}
cp Qpow(cp x , int p)
{
    cp res = 1;
    while(p)
    {
        if(p & 1)res = res * x;
        x = x * x;
        p >>= 1;
    }
    return res;
}
// == Main ==
int Legendre(int a){return Qpow(a , (mod - 1) / 2).a;}
pair<int , int> Cipolla(int a)
{
    if(a % mod == 0)return {0 , -1};
    if(Legendre(a) == mod - 1)return {-1 , -1};
    int res = 0;
    while(1)
    {
        int r = rnd() % mod;
        w2 = ((ll)r * r - a) % mod;
        if(w2 < 0)w2 += mod;
        if(w2 == 0)
        {
            res = r;
            break ;
        }
        if(Legendre(w2) == mod - 1)
        {
            res = Qpow(cp(r , 1) , (mod + 1) / 2).a;
            break ;
        }
    }
    if(res * 2 >= mod)res = mod - res;
    return {res , mod - res};
}
\end{minted}

\subsection{模素数 N 次剩余}

我们令 $x=g^c$ , $g$ 是 $p$ 的原根(我们一定可以找到这个 $g$ 和 $c$ ),问题转化为求解 $(g^c)^a \equiv b \pmod p$ . 稍加变换,得到
$$ (g^a)^c \equiv b \pmod p $$

于是就转换成了 BSGS 的基本模型了,可以在 $O(\sqrt p)$ 解出 $c$ ,这样可以得到原方程的一个特解 $x_0\equiv g^c\pmod p$ .



\section{min25 筛}

\subsection{简述}

应用范围:求 $\sum_{i=1}^n f(i)$,其中 $f(i)$ 是积性函数。

需要满足 $f(i)$ 在 $i$ 是质数时的取值是多项式。

时间复杂度:$\Theta(n^{1-\epsilon})$/$\Theta(\frac{n^{\frac{3}{4}}}{\log n})$。

主要想法是将 $f(i)$ 分成三个部分后求和:$i$ 是质数,$i$ 是合数,$i=1$。

先求出 $i$ 是质数部分的和。

为了方便起见,我们将 $f(i)$ 分解成若干 $x^k$ 之和,然后分别计算,这样我们要计算的函数就是**完全积性函数**了。

考虑埃氏筛的过程,设 $g_{n,k}$ 表示**筛了 $2 \sim n$ 中的数,用前 $k$ 个质数来筛,没有被筛掉的数的 $f'$ 值之和**。

设 $p_i$ 表示第 $i$ 个质数,考虑求出 $g$ 数组,首先容易发现如果 $p_k^2 > n$,$g_{n,k}=g_{n,k-1}$。

否则我们会筛掉一些数,由于是完全积性函数,容易得出转移:

$$g_{n,k}=g_{n,k-1}-f'(p_k) \times (g_{\lfloor \frac{n}{p_k} \rfloor,k-1}-\sum_{i=1}^{k-1}f'(p_i))$$

注意 $g$ 数组计算的答案是**“将没被筛掉的数也当成质数”时的答案**。

我们考虑计算答案,设 $s_{n,k}$ 表示**筛了 $2 \sim n$ 中的数,用前 $k$ 个质数来筛,没有被筛掉的数的 $f$ 值之和。**

现在这个 $s_{n,k}$ 求的就是真正的答案了,最终答案就是 $s_{n,0}+f(1)$。

设 $sp_i=\sum_{j=1}^i p_j$,$D$ 为最大的 $i$ 满足 $p_i^2 \le n$,容易得出转移:

$$s_{n,k}=g_{n,D}-sp_{k}+\sum_{p_i^e \le n \& i>k} f({p_i^e})(s_{\frac{n}{p_i^e},e}+[e>1])$$

然后暴力递归求出 $s$ 就行了,不需要记忆化。

\subsection{求 $1 \sim n$ 的素数个数}

\begin{minted}{c++}
#define int long long
#define id(x) ((x<=lim)?(id1[x]):(id2[n/(x)]))
#define rep(i,j,k) for(int i=j;i<=k;++i)
#define per(i,j,k) for(int i=j;i>=k;--i)
int const N=1e6+10;
int lim,prime[N],w[N],g[N],id1[N],id2[N],m;bool vis[N];
inline void init(){
    //预处理根号以内质数
    rep(i,2,lim){
        if (!vis[i]) prime[++prime[0]]=i;
        for (int j=1;j<=prime[0] && i*prime[j]<=lim;++j){
            vis[i*prime[j]]=1;
            if (i%prime[j]==0) break;
        }
    }
}
inline void solve(){
    int n;cin>>n,lim=sqrtl(n),init();
    for (int l=1,r;l<=n;l=r+1){
        r=n/(n/l),w[++m]=n/l,g[m]=w[m]-1;
        if (w[m]<=lim) id1[w[m]]=m;
        else id2[n/w[m]]=m;
        //预处理 g 数组第一维有可能的取值
    }
    rep(i,1,prime[0])
        for (int j=1;j<=m && prime[i]*prime[i]<=w[j];++j)
            g[j]-=g[id(w[j]/prime[i])]-(i-1);
    cout<<g[id(n)]<<'\n';
}
\end{minted}

\subsection{完整模板}

\begin{minted}{c++}
#define int long long
#define rep(i,j,k) for(int i=j;i<=k;++i)
#define per(i,j,k) for(int i=j;i>=k;--i)
#define id(x) ((x<=lim)?(id1[x]):(id2[n/(x)]))
#define add(x,y) (x=((x+y>=mod)?(x+y-mod):(x+y)))
//例:求 f(p^k)=p^k(p^k-1) 的前缀和
//发现 f(p^k)=p^{2k}-p^k
int const mod=1e9+7;
int n;
inline int f(int op,int x){
    x%=mod;
    if (!op) return x;
    return x*x%mod;
}
inline int F(int x){
    x%=mod;
    return x*(x-1)%mod;
}
inline int smf(int op,int x){
    x%=mod;
    if (!op) return (x*(x+1)/2)%mod;
    return x*(x+1)%mod*(2*x%mod+1)%mod*166666668%mod;
}
//smf -> f 前缀和
int const N=1e6+10;
int lim,prime[N],sp1[N],sp2[N],w[N],g1[N],g2[N],id1[N],id2[N],m;bool vis[N];
inline void init(){
    //预处理根号以内质数
    rep(i,2,lim){
        if (!vis[i]) prime[++prime[0]]=i;
        for (int j=1;j<=prime[0] && i*prime[j]<=lim;++j){
            vis[i*prime[j]]=1;
            if (i%prime[j]==0) break;
        }
    }
    rep(i,1,prime[0]){
        sp1[i]=sp1[i-1],add(sp1[i],f(0,prime[i]));
        sp2[i]=sp2[i-1],add(sp2[i],f(1,prime[i]));
    }
}
inline int S(int x,int y){
    if (prime[y]>=x) return 0;
    int ans=((g2[id(x)]+mod-g1[id(x)])%mod+mod-(sp2[y]-sp1[y]+mod)%mod)%mod;
    for (int i=y+1;i<=prime[0] && prime[i]*prime[i]<=x;++i)
        for (int e=1,gg=prime[i];gg<=x;++e,gg*=prime[i])
            add(ans,F(gg)*(S(x/gg,i)+(e>1))%mod);
    return ans;
}
inline void solve(){
    cin>>n,lim=sqrtl(n),init();
    for (int l=1,r;l<=n;l=r+1){
        r=n/(n/l),w[++m]=n/l;
        g1[m]=(smf(0,w[m])+mod-1)%mod;
        g2[m]=(smf(1,w[m])+mod-1)%mod;
        if (w[m]<=lim) id1[w[m]]=m;
        else id2[n/w[m]]=m;
        //预处理 g 数组第一维有可能的取值
    }
    rep(i,1,prime[0])
        for (int j=1;j<=m && prime[i]*prime[i]<=w[j];++j)
            add(g1[j],mod-f(0,prime[i])*(g1[id(w[j]/prime[i])]+mod-sp1[i-1])%mod),
            add(g2[j],mod-f(1,prime[i])*(g2[id(w[j]/prime[i])]+mod-sp2[i-1])%mod);
    cout<<(S(n,0)+1)%mod<<'\n';
}
\end{minted}{c++}


\section{Powerful Number 筛}

Powerful Number 是形如 $a^2b^3$ 的数(即不含幂次 $=1$ 的质因子),记 $1\dots n$ 的 Powerful Number 的集合为 $\mathrm{PN}(n)$。其大小是 $\Theta(\sqrt{n})$ 级别的。

构造一个积性函数 $g$,使得 $\forall p \in P, f(p) = g(p)$,称为素数拟合。

找到一个函数 $h$,使得 $f = g * h$($*$ 为狄利克雷卷积),由 $f(p)=g(p)$,可得 $h$ 只在 Powerful Number 处有值。推式子可得:

$$
F(n) = \sum\limits_{i=1}^n f(i) = \sum_{d\in\mathrm{PN}(n)}h(d)G(\left\lfloor\frac{n}{d}\right\rfloor)
$$

现在只需快速求出 $\mathrm{PN}(n)$、$h$ 在 $\mathrm{PN}(n)$ 上的值和 $G(n)$ 在 $\{\lfloor n/d\rfloor | d \in \mathrm{PN}(n)\}$ 上的值即可。

要求 $f(p^k),g(p^k)$ 好求,且 $G$($g$ 的前缀和)能杜教筛。

时间复杂度为求 $G$ 的复杂度。

\begin{minted}{c++}
// == Preparations ==
typedef __int128 i128;
const int N = 5e6 + 5 , M = 1e4 + 5 , LG = 34;
ll n , t , hp[M][LG] , tot , h[N] , pn[N];
int m , mm , pr[N]; bool b[N];
void Init(const int n = N - 5);// Init prime and prefix sum of g
ll G(ll n);// Du Sieve
// == Main ==
void Dfs(int x , ll v , ll hv)
{
    if(x == mm)return ;
    ll p = pr[x] , pk = p * p;
    if((i128)v * pk <= n)Dfs(x + 1 , v , hv);
    for(int i = 2 ; (i128)v * pk <= n ; i++ , pk *= p)
    {
        tot++ , pn[tot] = v * pk;
        h[tot] = hv * hp[x][i] % MOD;
        Dfs(x + 1 , v * pk , h[tot]);
    }
}
ll Solve(ll _n)
{
    Init(); n = _n , t = sqrt(n);
    for(mm = 1 ; pr[mm] <= t ; mm++);// need max prime > sqrt n
    for(int i = 1 ; i <= mm ; i++)
    {
        ll p = pr[i] , v = p; hp[i][0] = 1;
        for(int j = 1 ; v <= n ; j++ , v *= p)
        {
            ll g = p * (p - 1) % MOD; 
            hp[i][j] = v % MOD * ((v - 1) % MOD) % MOD;// f(p^j)
            for(int k = 1 ; k <= j ; k++)
            {
                hp[i][j] = (hp[i][j] - g * hp[i][j - k]) % MOD;// - g(p^k)*h(p^{j-k})
                g = g * p % MOD * p % MOD;
            }
            //h(p^j) = f(p^j) - \sum_{k=1^j}g(p^k)*h(p^{j-k})
        }
    }
    pn[1] = h[1] = tot = 1; 
    Dfs(1 , 1 , 1);
    ll ans = 0;
    for(int i = 1 ; i <= tot ; i++)
        ans = (ans + h[i] * G(n / pn[i])) % MOD;
    return (ans + MOD) % MOD;
}
\end{minted}

\section{组合数的一些有用的公式}

范德蒙德卷积及其推论:

$$
\begin{aligned}
\sum_{i=0}^k \binom{n}{i}\binom{m}{k-i}=\binom{n+m}{k} \\
\sum_{i=-r}^s \binom{n}{r+i}\binom{m}{s-i}=\binom{n+m}{r+s} \\
\sum_{i=1}^n \binom{n}{i}\binom{n}{i-1}=\binom{2n}{n-1} \\
\sum_{i=0}^n \binom{n}{i}^2=\binom{2n}{n} \\
\sum_{i=0}^m \binom{n}{i}\binom{m}{i}=\binom{n+m}{m}
\end{aligned}
$$

用来容斥的东西:

$$\sum_{i=0}^n (-1)^i \binom{n}{i}=[n=0]$$

组合数带权和(对于 $i^k$ 可以得出是关于 $n$ 的 $k$ 次多项式与 $2^{n-k}$ 的乘积):

$$\sum_{i=0}^n i \binom{n}{i}=n2^{n-1}$$

$$\sum_{i=0}^n i^2 \binom{n}{i}=n(n+1)2^{n-2}$$

上指标求和:

$$\sum_{i=m}^n \binom{i}{m}=\binom{n+1}{m+1}$$

三项式版恒等式:

$$\binom{n}{m}\binom{m}{k}=\binom{n}{k}\binom{n-k}{m-k}$$

平行恒等式:

$$\sum_{i=0}^n \binom{m+i}{i}=\binom{m+n+1}{n}$$

上指标卷积:

$$\sum_{i=0}^n \binom{i}{a}\binom{n-i}{b}=\binom{n+1}{a+b+1}$$

二项式反演:

$$f_n=\sum_{i=0}^n \binom{n}{i}g_i \leftrightarrow g_n=\sum_{i=0}^n (-1)^{n-i} \binom{n}{i} f_i$$

$$f_n=\sum_{i=n}^m \binom{i}{n} g_i \leftrightarrow g_n=\sum_{i=n}^m (-1)^{i-n}\binom{i}{n}f_i$$

多项式定理:

设 $n$ 为正整数,$x_i$ 为实数,则:

$$(x_1+x_2+...+x_t)^n=\sum_{\text{满足 $n_1+n_2+...+n_t=n$ 的非负整数解}} \binom{n}{n_1,n_2,...,n_t}x_1^{n_1}x_2^{n_2}\dots x_t^{n_t}$$

其中的 $\binom{n}{n_1,n_2,...,n_t}$ 是多重集的排列数,取 $x_1=x_2=...=x_t=1$,得到 $\sum \binom{n}{n_1,n_2,...,n_t}=t^n$。

错位排列:

$$D_n=\sum_{k=0}^n(-1)^k\binom{n}{k}(n-k)!$$

卡特兰数:

$$H_n=\frac{1}{n+1}\binom{2n}{n}=\frac{(2n)!}{n!(n+1)!}$$

$$H_n=\binom{2n}{n}-\binom{2n}{n-1}$$

$$H_n=\frac{H_{n-1}(4n-2)}{n+1}$$



%
%% Chapter ? 多项式
\chapter{多项式}

\section{拉格朗日插值法}

$$
f(k) = \sum\limits_{i = 0}^n y_i \prod\limits_{j \neq i} \dfrac{k - x_j}{x_i - x_j}
$$

取值连续时的优化:

当 $x_i = i$ 时可以优化成 $O(n)$。

$$
\begin{aligned}
f(k) &= \sum\limits_{i = 0}^n y_i \prod\limits_{j \neq i} \dfrac{k - x_j}{x_i - x_j} \\
&= \sum\limits_{i = 0}^n y_i \prod\limits_{j \neq i} \dfrac{k - j}{i - j}
\end{aligned}
$$

令 $pre_i = \prod\limits_{j = 0}^i (k - j), suf_i = \prod\limits_{j = i}^n (k - j)$:

$$
f(k) = \sum\limits_{i = 0}^n y_i \dfrac{pre_{i - 1} \times suf_{i + 1}}{i! \times (n - i)!} \times (-1)^{n - i}
$$

重心拉格朗日插值:

$$
\begin{aligned}
f(k) &= \sum\limits_{i = 0}^n y_i \prod\limits_{i \neq j} \dfrac{k - x_j}{x_i - x_j} \\
&= \sum\limits_{i = 0}^n y_i \dfrac{\prod\limits_{j \neq i} k - x_j}{\prod\limits_{j \neq i} x_i - x_j} \\
&= \sum\limits_{i = 0}^n \dfrac{y_i}{k - x_i} \dfrac{\prod\limits_{j = 0}^n k - x_j}{\prod\limits_{j \neq i} x_i - x_j} \\
\end{aligned}
$$

设 $g(k) = \prod\limits_{i = 0}^n k - x_i, t_i = \dfrac{y_i}{\prod\limits_{j \neq i} x_i - x_j}$。

则

$$
\begin{aligned}
f(k) &= \sum\limits_{i = 0}^n \dfrac{g(k)t_i}{k - x_i} \\
&= g(k) \sum\limits_{i = 0}^n \dfrac{t_i}{k - x_i}
\end{aligned}
$$

于是每次插入只需要更新之前的所有 $t_i$ 并计算当前点的 $t_i$ 即可,时间复杂度为 $O(n)$。

由于 $g(k)$ 和 $\sum\limits_{i = 0}^k \dfrac{t_i}{k - x_i}$ 都可以 $O(n)$ 求得,故询问复杂度也是 $O(n)$。


\section{牛顿迭代}

用于解决下列问题:

\begin{tcolorbox}
已知函数 $G$ 且 $G(F(x)) = 0$,求多项式 $F$(${}\bmod x^n$)。
\end{tcolorbox}

\textbf{结论}:

$$
F(x) = F_*(x) - \frac{G(F_*(x))}{G'(F_*(x))} \pmod{x^n}
$$

其中 $F_*(x)$ 为做到 $x^{n / 2}$ 时的答案。

\section{FFT}

\begin{minted}{c++}
// == Preparations ==
struct complex {
    double a, b;

    complex() = default;
    complex(double _a, double _b): a(_a), b(_b) {}
    complex operator+(const complex &x) const {return complex(a + x.a, b + x.b);}
    complex operator-(const complex &x) const {return complex(a - x.a, b - x.b);}
    complex operator*(const complex &x) const {return complex(a * x.a - b * x.b, a * x.b + b * x.a);}
    complex operator/(const complex &x) const {
        double t = b * b + x.b * x.b;
        return complex((a * x.a + b * x.b) / t, (b * x.a - a * x.b) / t);
    }
    complex &operator+=(const complex &x) {return *this = *this + x;}
    complex &operator-=(const complex &x) {return *this = *this - x;}
    complex &operator*=(const complex &x) {return *this = *this * x;}
    complex &operator/=(const complex &x) {return *this = *this / x;}
};
// == Main ==
void FFT(vector<complex> &f, int flag) const {
    int n = f.size();
    vector<int> swp(n);
    for (int i = 0; i < n; i++) {
        swp[i] = swp[i >> 1] >> 1 | ((i & 1) * (n >> 1));
        if (i < swp[i]) std::swap(f[i], f[swp[i]]);
    }
    for (int mid = 1; mid < n; mid <<= 1) {
        complex w1(cos(pi / mid), flag * sin(pi / mid));
        for (int i = 0; i < n; i += mid << 1) {
            complex w(1, 0);
            for (int j = 0; j < mid; j++, w *= w1) {
                complex x = f[i + j], y = w * f[i + mid + j];
                f[i + j] = x + y, f[i + mid + j] = x - y;
            }
        }
    }
    return;
}
\end{minted}

\section{常用 NTT 模数及其原根}

\begin{center}
    \begin{tabular}{c|c|c}
        % \specialrule{2pt}{0em}{0em}
        模数 & 原根 & 分解 \\
        \hline
        % \specialrule{1pt}{0em}{0em}
        $167772161$ & $3$ & $5 \times 2^{25} + 1$ \\
        \hline
        $469762049$ & $3$ & $7 \times 2^{26} + 1$ \\
        \hline
        $998244353$ & $3$ & $119 \times 2^{23} + 1$ \\
        \hline
        $1004535809$ & $3$ & $479 \times 2^{21} + 1$ \\
        \hline
        $2013265921$ & $31$ & $15 \times 2^{27} + 1$ \\
        \hline
        $2281701377$ & $3$ & $17 \times 2^{27} + 1$ \\
        % \specialrule{2pt}{0em}{0em}
    \end{tabular}
\end{center}

\input{chapters/poly/poly}

\section{集合幂级数}

\subsection{模板}

\begin{minted}{c++}
// == Preparations ==
const int mod = 998244353, N = 20;

int power(int a, int b) {
    int ans = 1;
    while (b) {
        if (b & 1) ans = (long long)ans * a % mod;
        a = (long long)a * a % mod;
        b >>= 1;
    }
    return ans % mod;
}
// == Main ==
// 需要保证除了 [__builtin_popcount(S)][S] 之外的其他位置没有值,需要保证目标数组为空
inline int &inc(int &x, const int &y) {if (int(unsigned(x += y) - mod) >= 0) x -= mod; return x;}
inline int &dec(int &x, const int &y) {return x = x - y < 0 ? x - y + mod : x - y;}
inline int &getval(int f[N + 1][1 << N], int S) {return f[__builtin_popcount(S)][S];}
void FWT(int *f, int n) {
    for (int mid = 1; mid < 1 << n; mid <<= 1)
        for (int i = 0; i < 1 << n; i += mid << 1)
            for (int j = 0; j < mid; j++) inc(f[i + mid + j], f[i + j]);
    return;
}
void IFWT(int *f, int n) {
    for (int mid = 1; mid < 1 << n; mid <<= 1)
        for (int i = 0; i < 1 << n; i += mid << 1)
            for (int j = 0; j < mid; j++) dec(f[i + mid + j], f[i + j]);
    return;
}
void mul(int a[N + 1][1 << N], int b[N + 1][1 << N], int ans[N + 1][1 << N], int n) {
    for (int i = 0; i <= n; i++) FWT(a[i], n);
    for (int i = 0; i <= n; i++) FWT(b[i], n);
    for (int i = 0; i <= n; i++)
        for (int j = 0; j <= n - i; j++)
            for (int k = 0; k < 1 << n; k++)
                ans[i + j][k] = (ans[i + j][k] + (long long)a[i][k] * b[j][k]) % mod;
    for (int i = 0; i <= n; i++) IFWT(a[i], n);
    for (int i = 0; i <= n; i++) IFWT(b[i], n);
    for (int i = 0; i <= n; i++) IFWT(ans[i], n);
    return;
}
void inv(int f[N + 1][1 << N], int g[N + 1][1 << N], int n) {
    static int tmp[21][1 << 20];
    for (int i = 0; i <= n; i++)
        for (int j = 0; j < n; j++) FWT(f[i] + (1 << j), j);
    g[0][0] = power(f[0][0], mod - 2);
    for (int i = 1; i <= n; i++) {
        for (int j = 0; j <= i; j++)
            for (int k = 0; k <= i - j; k++)
                for (int l = 0; l < 1 << (i - 1); l++)
                    tmp[j + k][l] = (tmp[j + k][l] +
                        (long long)(mod - f[j][l | 1 << (i - 1)]) * g[k][l]) % mod;
        for (int j = 0; j <= i; j++)
            for (int k = 0; k <= i - j; k++)
                for (int l = 0; l < 1 << (i - 1); l++)
                    g[j + k][l | 1 << (i - 1)] = (g[j + k][l | 1 << (i - 1)] +
                        (long long)tmp[j][l] * g[k][l]) % mod;
        for (int j = 0; j <= i; j++)
            for (int k = 0; k < 1 << (i - 1); k++) inc(g[j][k | 1 << (i - 1)], g[j][k]), tmp[j][k] = 0;
    }
    for (int i = 0; i <= n; i++) IFWT(g[i], n);
    for (int i = 0; i <= n; i++)
        for (int j = 0; j < n; j++) IFWT(f[i] + (1 << j), j);
    return;
}
void exp(int f[N + 1][1 << N], int g[N + 1][1 << N], int n) {
    for (int i = 0; i <= n; i++)
        for (int j = 0; j < n; j++) FWT(f[i] + (1 << j), j);
    g[0][0] = 1;
    for (int i = 1; i <= n; i++) {
        for (int j = 0; j <= i; j++)
            for (int k = 0; k <= i - j; k++)
                for (int l = 0; l < 1 << (i - 1); l++)
                    g[j + k][l | 1 << (i - 1)] = (g[j + k][l | 1 << (i - 1)] +
                        (long long)f[j][l | 1 << (i - 1)] * g[k][l]) % mod;
        for (int j = 0; j <= i; j++)
            for (int k = 0; k < 1 << (i - 1); k++) inc(g[j][k | 1 << (i - 1)], g[j][k]);
    }
    for (int i = 0; i <= n; i++) IFWT(g[i], n);
    for (int i = 0; i <= n; i++)
        for (int j = 0; j < n; j++) IFWT(f[i] + (1 << j), j);
    return;
}
void ln(int f[N + 1][1 << N], int g[N + 1][1 << N], int n) {
    FWT(f[0], 1);
    for (int j = 0; j < n; j++) FWT(f[0] + (1 << j), j);
    for (int i = 1; i <= n; i++) {
        FWT(f[i], i - 1);
        for (int j = i - 1; j < n; j++) FWT(f[i] + (1 << j), j);
    }
    for (int i = 1; i <= n; i++) {
        for (int j = 0; j <= i; j++) {
            for (int k = 0; k < 1 << (i - 1); k++) inc(g[j][k | 1 << (i - 1)], f[j][k | 1 << (i - 1)]);
            for (int k = 1; k <= i - j; k++)
                for (int l = 0; l < 1 << (i - 1); l++)
                    g[j + k][l | 1 << (i - 1)] = (g[j + k][l | 1 << (i - 1)] +
                        (long long)(mod - g[j][l | 1 << (i - 1)]) * f[k][l]) % mod;
            for (int k = 0; k < 1 << (i - 1); k++) inc(f[j][k | 1 << (i - 1)], f[j][k]);
        }
    }
    for (int i = 0; i <= n; i++)
        for (int j = 0; j < n; j++) IFWT(g[i] + (1 << j), j);
    for (int i = 0; i <= n; i++) IFWT(f[i], n);
    return;
}
\end{minted}

\subsection{FWTxor}

\begin{minted}{c++}
void FWTxor(int a[], int op) {
    for (int l = 2; l <= n; l <<= 1)
        for (int i = 0; i < n; i += l)
            for (int j = 0; j < l / 2; j++) {
                int p = (a[i + j] + a[i + j + l / 2]) % mod, d = (a[i + j] - a[i + j + l / 2] + mod) % mod;
                a[i + j] = (long long)p * op % mod;
                a[i + j + l / 2] = (long long)d * op % mod;
            }
    return;
}
\end{minted}
%
%% Chapter ? 杂项
\chapter{杂项}

\section{自定义哈希}

\subsection{splitmix 和仿函数}

\begin{minted}{c++}
// == Preparations ==
#include <chrono>
#include <random>

mt19937 gen(chrono::system_clock::now().time_since_epoch().count());
mt19937_64 genll(chrono::system_clock::now().time_since_epoch().count());
// == Main ==
unsigned splitmix32(unsigned x) {
    unsigned z = (x += 0x9E3779B9);
    z = (z ^ (z >> 16)) * 0x85ebca6b;
    z = (z ^ (z >> 13)) * 0xc2b2ae35;
    return z ^ (z >> 16);
}

unsigned long long splitmix64(unsigned long long x) {
    unsigned long long z = (x += 0x9e3779b97f4a7c15);
    z = (z ^ (z >> 30)) * 0xbf58476d1ce4e5b9;
    z = (z ^ (z >> 27)) * 0x94d049bb133111eb;
    return z ^ (z >> 31);
}

struct custom_hash { // custom hash for unsigned long long
    size_t operator()(unsigned long long x) const {
        return splitmix64(x + genll());
    }
};

struct custom_hash { // custom hash for pair<unsigned long long, unsigned long long>
    size_t operator()(pair<unsigned long long, unsigned long long> x) const {
        return (splitmix64(x.first + genll()) << 1) ^ splitmix64(x.second + genll());
    }
};
\end{minted}

\subsection{xorshift}

\begin{minted}{c++}
// == Main ==
unsigned long long xorshift(unsigned long long x) { x ^= x << 13; x ^= x >> 7; x ^= x << 17; return x; }
unsigned xorshift(unsigned x) { x ^= x << 13; x ^= x >> 17; x ^= x << 5; return x; }
\end{minted}

\subsection{手写哈希表}

\begin{minted}{c++}
// == Main ==
template<typename ValType , typename KeyType>
struct HashTable
{
    static const int M = 13075 , P = 13003; // P < M and P is between N^1.1 to N^1.7
    struct Node{KeyType key; ValType v; int nxt;}e[M];
    int tot , head[P];
    inline void Add(int u , const KeyType& k , const ValType &v){e[++tot] = {k , v , head[u]} , head[u] = tot;}
    inline int hash(const KeyType& x){return x;}
    ValType& operator[](const KeyType& x)
    {
        for(int i = head[hash(x) % P] ; i ; i = e[i].nxt)
            if(e[i].key == x)return e[i].v;
        Add(hash(x) % P , x , ValType());
        return e[tot].v;
    }
    ValType at(const KeyType& x)
    {
        for(int i = head[hash(x) % P] ; i ; i = e[i].nxt)
            if(e[i].key == x)return e[i].v;
        return ValType();
    }
    void clear()
    {
        for(int i = 1 ; i <= tot ; i++)
            head[hash(e[i].key) % P] = 0 , e[i] = {};
        tot = 0;
    }
};
\end{minted}

\section{取模类}

\begin{minted}{c++}
// == Main ==
struct mint {
    static const int mod = 998244353;
    int v;

    mint() = default;
    mint(int _v): v((_v % mod + mod) % mod) {}
    explicit operator int() const {return v;}
    mint operator+(const mint &x) const {return v + x.v - (v + x.v < mod ? 0 : mod);}
    mint &operator+=(const mint &x) {return *this = *this + x;}
    mint operator-(const mint &x) const {return v - x.v + (v - x.v >= 0 ? 0 : mod);}
    mint &operator-=(const mint &x) {return *this = *this - x;}
    mint operator*(const mint &x) const {return (long long)v * x.v % mod;}
    mint &operator*=(const mint &x) {return *this = *this * x;}
    mint inv() const {
        mint a(*this), ans(1);
        int b(mod - 2);
        while (b) {
            if (b & 1) ans *= a;
            a *= a;
            b >>= 1;
        }
        return ans;
    }
    mint operator/(const mint &x) const {return *this * x.inv();}
    mint &operator/=(const mint &x) {return *this = *this / x;}
    mint operator-() {return mint(-v);}
};
\end{minted}

\subsection{Barrett 约减}

当模数不固定时可以加速。

用法:在构造函数中传模数,使用方法为 \lstinline|F.reduce(x)|,其中 $x$ 是需要取模的数。

\begin{minted}{c++}
// == Main ==
struct Barrett {
    unsigned long long b, m;
    Barrett(unsigned long long b = 2): b(b), m((__uint128_t(1) << 64) / b) {}
    unsigned long long reduce(long long x) {
        unsigned long long r = (__uint128_t(x + b) * m) >> 64;
        unsigned long long q = (x + b) - b * r;
        return q >= b ? q - b : q;
    }
} F;
\end{minted}

\section{快速模乘算法}

\begin{minted}{c++}
// == Preparations ==
typedef __float128 f128;
typedef __int128 i128;
typedef __uint128_t u128;
// == Main ==
inline i128 mul(i128 x , i128 y , i128 p)
{
    i128 z = (f128) x / p * y;
    i128 res = (u128)x * y - (u128)z * p;
    return (res + p) % p;
}
\end{minted}

\section{对拍脚本}

\begin{minted}{bash}
#!/usr/bin/bash

declare -i num=0

while [ true ]; do
    ./mkdata > in.txt
    time ./mine < in.txt > out.txt
    ./correct < in.txt > ans.txt
    diff out.txt ans.txt
    if [ $? -ne 0 ]; then
        echo "WA"
        break
    fi
    num=num+1
    echo "Passed $num tests."
done
\end{minted}

\section{VS Code 配置}

\subsection{User Tasks}

\begin{minted}{json}
{
    // See https://go.microsoft.com/fwlink/?LinkId=733558
    // for the documentation about the tasks.json format
    "version": "2.0.0",
    "tasks": [
        {
            "type": "shell",
            "label": "My C++ Runner",
            "detail": "Build and Run Current C++ Program",
            "command": [ // 三个编译方式保留一个即可。
                "clear",
                "&&",
                "g++ ${file} -o ${fileDirname}/${fileBasenameNoExtension} -std=c++14 -Wall -Wextra && echo '== Normal =='",
                "g++ ${file} -o ${fileDirname}/${fileBasenameNoExtension} -std=c++14 -Wall -Wextra -O2 && echo '== O2 =='",
                "g++ ${file} -o ${fileDirname}/${fileBasenameNoExtension} -std=c++14 -Wall -Wextra -fsanitize=undefined,address && echo '== UB Check =='",
                "&&",
                "gnome-terminal -- bash -c \"ulimit -s 524288; time ${fileDirname}/${fileBasenameNoExtension}; read -p 'Press ENTER to continue...'; exit\""
            ],
            "problemMatcher": [ // 非必要
                "$gcc"
            ],
            "group": { // 非必要
                "kind": "build",
                "isDefault": true
            },
            "presentation": { // 非必要
                "showReuseMessage": false
            }
        }
    ]
}
\end{minted}

For Windows:

命令面板输入 \lstinline|Terminal: Select Default Profile|,选择 cmd。

\begin{minted}{json}
{
    // See https://go.microsoft.com/fwlink/?LinkId=733558
    // for the documentation about the tasks.json format
    "version": "2.0.0",
    "tasks": [
        {
            "label": "tang loong wenmo",
            "type": "shell",
            "command": [
                "g++ ${file} -o ${fileDirname}/${fileBasenameNoExtension} -std=c++14 -O2 -Wl,--stack=2147483647",
                "&&",
                "start cmd /c \"${fileDirname}/${fileBasenameNoExtension} & echo.&pause\""
            ],
            "options": {
                "cwd": "${fileDirname}"
            },
            "group": {
                "kind": "build",
                "isDefault": true
            }
        }
    ]
}
\end{minted}

\subsection{设置}

\begin{itemize}
    \item 不允许 Enter 进行代码补全。(\lstinline|"editor.acceptSuggestionOnEnter": "off"|)
\end{itemize}

\end{document}
