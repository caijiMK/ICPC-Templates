\section{取模类}

\begin{minted}{c++}
// == Main ==
struct mint {
    static const int mod = 998244353;
    int v;

    mint() = default;
    mint(int _v): v((_v % mod + mod) % mod) {}
    explicit operator int() const {return v;}
    mint operator+(const mint &x) const {return v + x.v - (v + x.v < mod ? 0 : mod);}
    mint &operator+=(const mint &x) {return *this = *this + x;}
    mint operator-(const mint &x) const {return v - x.v + (v - x.v >= 0 ? 0 : mod);}
    mint &operator-=(const mint &x) {return *this = *this - x;}
    mint operator*(const mint &x) const {return (long long)v * x.v % mod;}
    mint &operator*=(const mint &x) {return *this = *this * x;}
    mint inv() const {
        mint a(*this), ans(1);
        int b(mod - 2);
        while (b) {
            if (b & 1) ans *= a;
            a *= a;
            b >>= 1;
        }
        return ans;
    }
    mint operator/(const mint &x) const {return *this * x.inv();}
    mint &operator/=(const mint &x) {return *this = *this / x;}
    mint operator-() {return mint(-v);}
};
\end{minted}

\subsection{Barrett 约减}

当模数不固定时可以加速。

用法:在构造函数中传模数,使用方法为 \lstinline|F.reduce(x)|,其中 $x$ 是需要取模的数。

\begin{minted}{c++}
// == Main ==
struct Barrett {
    unsigned long long b, m;
    Barrett(unsigned long long b = 2): b(b), m((__uint128_t(1) << 64) / b) {}
    unsigned long long reduce(long long x) {
        unsigned long long r = (__uint128_t(x + b) * m) >> 64;
        unsigned long long q = (x + b) - b * r;
        return q >= b ? q - b : q;
    }
} F;
\end{minted}