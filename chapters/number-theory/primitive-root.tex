\section{原根}

\subsection{阶}

\begin{tcolorbox}
\textbf{定义}:满足同余式 $a^n \equiv 1 \pmod m$ 的最小正整数 $n$ 存在,这个 $n$ 称作 $a$ 模 $m$ 的阶,记作 $\delta_m(a)$ 或 $\operatorname{ord}_m(a)$。
\end{tcolorbox}

\begin{tcolorbox}
\textbf{性质 1}:$a,a^2,\cdots,a^{\delta_m(a)}$ 模 $m$ 两两不同余。
\end{tcolorbox}

\begin{tcolorbox}
\textbf{性质 2}:若 $a^n \equiv 1 \pmod m$ ,则 $\delta_m(a)\mid n$。
\end{tcolorbox}

\begin{tcolorbox}
\textbf{性质 2 推论}:若 $a^p\equiv a^q\pmod m$ ,则有 $p\equiv q\pmod{\delta_m(a)}$。
\end{tcolorbox}

\begin{tcolorbox}
\textbf{性质 3}:

设 $m\in\mathbf{N}^{*}$ , $a,b\in\mathbf{Z}$ , $(a,m)=(b,m)=1$ ,则

$$\delta_m(ab)=\delta_m(a)\delta_m(b)$$

的充分必要条件是

$$\left(\delta_m(a), \delta_m(b)\right)=1$$
\end{tcolorbox}

\begin{tcolorbox}
\textbf{性质 4}:

设 $k \in \mathbf{N}$ , $m$ 为正整数, $a\in\mathbf{Z}$ , $(a,m)=1$ ,则
$$\delta_m(a^k)=\dfrac{\delta_m(a)}{\left(\delta_m(a),k\right)}$$
\end{tcolorbox}

\subsection{原根}


\begin{tcolorbox}
\textbf{定义}:

设 $m$ 为正整数, $g$ 为整数。 若 $(g,m)=1$ ,且 $\delta_m(g)=\varphi(m)$ ,则称 $g$ 为模 $m$ 的原根。

即 $g$ 满足
$\delta_m(g) = \varphi(m)$。 当 $m$ 是质数时,我们有 $g^i \bmod m,0 < i < m$ 的结果互不相同。
\end{tcolorbox}


\begin{tcolorbox}
\textbf{原根判定定理}:

设 $m \ge 3, (g,m)=1$ ,则 $g$ 是模 $m$ 的原根的充要条件是,对于 $\varphi(m)$ 的每个素因数 $p$ ,都有
$g^{\frac{\varphi(m)}{p}}\not\equiv 1\pmod m$。
\end{tcolorbox}


\begin{tcolorbox}
\textbf{原根个数}:

若一个数 $m$ 有原根,则它原根的个数为 $\varphi(\varphi(m))$。
\end{tcolorbox}


\begin{tcolorbox}
\textbf{原根存在定理}:

一个数 $m$ 存在原根当且仅当 $m=2,4,p^{\alpha},2p^{\alpha}$ ,其中 $p$ 为奇素数, $\alpha\in \mathbf{N}^{*}$。
\end{tcolorbox}

\begin{minted}{c++}
// == Preparations ==
const int N = 1e6 + 5; 
int m , pr[N] , phi[N] , vis[N] , mod; bitset<N>b;
void Sieve(int n = N - 5);
int Qpow(int x , int p , int mod);
// == Main ==
vector<int> PrimitiveRoot()
{
    int p , d; vector<int>pfac , g;
    cin >> p >> d; mod = p;
    if(!vis[p]){return g;} // whether exist primitive root
    for(int i = 1 ; i <= m && pr[i] <= phi[p] ; i++)
        if(phi[p] % pr[i] == 0)pfac.push_back(pr[i]);
    for(int a = 1 ; a < p ; a++) // minimal primitive root is O(n^{0.25+eps})
    {
        if(__gcd(a , p) != 1)continue ;
        bool ok = 1;
        for(int x : pfac)if(Qpow(a , phi[p] / x , mod) == 1){ok = 0; break ;}
        if(!ok)continue ;
        g.push_back(a); break ;
    }
    for(int i = 2 ; i < phi[p] ; i++)
        if(__gcd(i , phi[p]) == 1)g.push_back(Qpow(g[0] , i , mod));
    sort(g.begin() , g.end());
    return g;
}
\end{minted}