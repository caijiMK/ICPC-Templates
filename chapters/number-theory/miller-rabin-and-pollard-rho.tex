\section{Miller Rabin 和 Pollard Rho}

\subsection{Miller Rabin}

\begin{minted}{c++}
// == Preparations ==
const int prime[] = {2, 3, 5, 7, 9, 11, 13, 17, 19, 23, 29, 31, 37};

long long power(long long a, long long b, long long mod) {
    long long ans = 1;
    while (b) {
        if (b & 1) ans = (__int128)ans * a % mod;
        a = (__int128)a * a % mod;
        b >>= 1;
    }
    return ans % mod;
}
// == Main ==
inline int Miller_Rabin(long long n) {
    if (n == 1) return 0;
    if (n == 2) return 1;
    if (n % 2 == 0) return 0;
    long long u = n - 1, t = 0;
    while (u % 2 == 0) u /= 2, t++;
    for (int i = 0; i < 12; i++) {
        if (prime[i] % n == 0) continue;
        long long x = power(prime[i] % n, u, n);
        if (x == 1) continue;
        int flag = 0;
        for (int j = 1; j <= t; j++) {
            if (x == n - 1) {flag = 1; break;}
            x = (__int128)x * x % n;
        }
        if (!flag) return 0;
    }
    return 1;
}
\end{minted}

\subsection{Pollard Rho}

\begin{minted}{c++}
// == Preparations ==
#include <chrono>
#include <random>

mt19937_64 gen(chrono::system_clock::now().time_since_epoch().count());
/*
Miller Rabin
*/
// == Main ==
long long Pollard_Rho(long long n) {
    long long s = 0, t = 0, c = gen() % (n - 1) + 1;
    for (int goal = 1; ; goal <<= 1, s = t) {
        long long val = 1;
        for (int step = 1; step <= goal; step++) {
            t = ((__int128)t * t + c) % n;
            val = (__int128)val * abs(t - s) % n;
            if (!val) return n;
            if (step % 127 == 0) {
                long long d = __gcd(val, n);
                if (d > 1) return d;
            }
        }
        long long d = __gcd(val, n);
        if (d > 1) return d;
    }
}
void factor(long long n) {
    if (n < ans) return ;
    if (n == 1 || Miller_Rabin(n)) {
        // n = 1 或 n 是一个质因子。
        // ...
    }
    long long p;
    do p = Pollard_Rho(n);
    while (p == n);
    while (n % p == 0) n /= p;
    factor(n), factor(p);
    return ;
}
\end{minted}