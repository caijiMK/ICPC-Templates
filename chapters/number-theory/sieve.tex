\section{筛法}

\subsection{杜教筛}

\subsubsection{狄利克雷卷积}

定义两个数论函数 $f$ 和 $g$ 的狄利克雷卷积为:$(f*g)(n)=\sum_{d|n}f(d) \times g(\frac{n}{d})$。

一些常见的卷积式:

$$\mu*I=\epsilon$$

$$\phi*I=id$$

\subsubsection{杜教筛}

$f$ 是一个积性函数,我们需要求 $f$ 的前缀和。

考虑构造两个积性函数 $h,g$,满足 $h=f*g$。

设 $S(n)=\sum_{i=1}^n f(i)$。

套路式:

$$\sum_{i=1}^n h(i)=\sum_{i=1}^n\sum_{d|i}g(d) \times f(\frac{i}{d})$$

$$\sum_{i=1}^n h(i)=\sum_{i=1}^n\sum_{d=1}^{\lfloor\frac{n}{i}\rfloor} g(i) \times f(d)$$

$$\sum_{i=1}^n h(i)=\sum_{i=1}^n g(i) \times S(\lfloor \frac{n}{i} \rfloor)$$

然后将右边式子的第一项提出来,可以得到:

$$\sum_{i=1}^n h(i)=g(1) \times S(n)+\sum_{i=2}^n g(i) \times S(\lfloor \frac{n}{i} \rfloor)$$

$$g(1) \times S(n)=\sum_{i=1}^n h(i)-\sum_{i=2}^n g(i) \times S(\lfloor \frac{n}{i} \rfloor)$$

经过复杂度分析后得出:如果能快速求出 $h$ 和 $g$ 的前缀和,求 $S(n)$ 的过程递归处理即可得到 $\Theta(n^{\frac{2}{3}})$ 的复杂度,\textbf{需要记忆化,并且先预处理出一部分值}。

\subsubsection{示例代码}

求 $\phi/\mu$ 的前缀和:

\begin{minted}{c++}
//A tree without skin will surely die.
//A man without face is invincible.
#include<bits/stdc++.h>
using namespace std;
#define int long long
int const N=1e7+10;
int const maxn=1e7;
int phi[N],prime[N],mu[N];
bitset<N>vis;
inline void init(){
    phi[1]=mu[1]=1;
    for (int i=2;i<=maxn;++i){
        if (!vis[i])
            prime[++prime[0]]=i,phi[i]=i-1,mu[i]=-1;
        for (int j=1;j<=prime[0] && prime[j]*i<=maxn;++j){
            vis[prime[j]*i]=1;
            if (i%prime[j]==0){
                phi[i*prime[j]]=phi[i]*prime[j];
                break;
            }
            phi[i*prime[j]]=phi[i]*phi[prime[j]];
            mu[i*prime[j]]=-mu[i];
        }
    }
    for (int i=2;i<=maxn;++i)
        phi[i]+=phi[i-1],mu[i]+=mu[i-1];
}
//mu*I=e
//f *g=h
unordered_map<int,int>mpmu,mpphi;
inline int summu(int n){
    if (n<=maxn) return mu[n];
    if (mpmu[n]) return mpmu[n];
    int ans=1;
    for (int l=2,r;l<=n;l=r+1)
        r=n/(n/l),ans-=(r-l+1)*summu(n/l);
    return mpmu[n]=ans;
}
//phi*I=id
//f  *g=h
inline int sumphi(int n){
    if (n<=maxn) return phi[n];
    if (mpphi[n]) return mpphi[n];
    int ans=n*(n+1)/2;
    for (int l=2,r;l<=n;l=r+1)
        r=n/(n/l),ans-=(r-l+1)*sumphi(n/l);
    return mpphi[n]=ans;
}
void solve(){
    int n;cin>>n;
    cout<<sumphi(n)<<' '<<summu(n)<<'\n';
}
signed main(){
    ios::sync_with_stdio(false);
    cin.tie(0),cout.tie(0);
    int t=1;
    cin>>t,init();
    while (t--) solve();
    return 0;
}
\end{minted}

\subsection{Powerful Number 筛}

Powerful Number 是形如 $a^2b^3$ 的数(即不含幂次 $=1$ 的质因子),记 $1\dots n$ 的 Powerful Number 的集合为 $\mathrm{PN}(n)$。其大小是 $\Theta(\sqrt{n})$ 级别的。

构造一个积性函数 $g$,使得 $\forall p \in P, f(p) = g(p)$,称为素数拟合。

找到一个函数 $h$,使得 $f = g * h$($*$ 为狄利克雷卷积),由 $f(p)=g(p)$,可得 $h$ 只在 Powerful Number 处有值。推式子可得:

$$
F(n) = \sum\limits_{i=1}^n f(i) = \sum_{d\in\mathrm{PN}(n)}h(d)G(\left\lfloor\frac{n}{d}\right\rfloor)
$$

现在只需快速求出 $\mathrm{PN}(n)$、$h$ 在 $\mathrm{PN}(n)$ 上的值和 $G(n)$ 在 $\{\lfloor n/d\rfloor | d \in \mathrm{PN}(n)\}$ 上的值即可。

要求 $f(p^k),g(p^k)$ 好求,且 $G$($g$ 的前缀和)能杜教筛。

时间复杂度为求 $G$ 的复杂度。

\begin{minted}{c++}
// == Preparations ==
typedef __int128 i128;
const int N = 5e6 + 5 , M = 1e4 + 5 , LG = 34;
ll n , t , hp[M][LG] , tot , h[N] , pn[N];
int m , mm , pr[N]; bool b[N];
void Init(const int n = N - 5);// Init prime and prefix sum of g
ll G(ll n);// Du Sieve
// == Main ==
void Dfs(int x , ll v , ll hv)
{
    if(x == mm)return ;
    ll p = pr[x] , pk = p * p;
    if((i128)v * pk <= n)Dfs(x + 1 , v , hv);
    for(int i = 2 ; (i128)v * pk <= n ; i++ , pk *= p)
    {
        tot++ , pn[tot] = v * pk;
        h[tot] = hv * hp[x][i] % MOD;
        Dfs(x + 1 , v * pk , h[tot]);
    }
}
ll Solve(ll _n)
{
    Init(); n = _n , t = sqrt(n);
    for(mm = 1 ; pr[mm] <= t ; mm++);// need max prime > sqrt n
    for(int i = 1 ; i <= mm ; i++)
    {
        ll p = pr[i] , v = p; hp[i][0] = 1;
        for(int j = 1 ; v <= n ; j++ , v *= p)
        {
            ll g = p * (p - 1) % MOD; 
            hp[i][j] = v % MOD * ((v - 1) % MOD) % MOD;// f(p^j)
            for(int k = 1 ; k <= j ; k++)
            {
                hp[i][j] = (hp[i][j] - g * hp[i][j - k]) % MOD;// - g(p^k)*h(p^{j-k})
                g = g * p % MOD * p % MOD;
            }
            //h(p^j) = f(p^j) - \sum_{k=1^j}g(p^k)*h(p^{j-k})
        }
    }
    pn[1] = h[1] = tot = 1; 
    Dfs(1 , 1 , 1);
    ll ans = 0;
    for(int i = 1 ; i <= tot ; i++)
        ans = (ans + h[i] * G(n / pn[i])) % MOD;
    return (ans + MOD) % MOD;
}
\end{minted}

\subsection{min25 筛}

\subsubsection{简述}

应用范围:求 $\sum_{i=1}^n f(i)$,其中 $f(i)$ 是积性函数。

需要满足 $f(i)$ 在 $i$ 是质数时的取值是多项式。

时间复杂度:$\Theta(n^{1-\epsilon})$/$\Theta(\frac{n^{\frac{3}{4}}}{\log n})$。

主要想法是将 $f(i)$ 分成三个部分后求和:$i$ 是质数,$i$ 是合数,$i=1$。

先求出 $i$ 是质数部分的和。

为了方便起见,我们将 $f(i)$ 分解成若干 $x^k$ 之和,然后分别计算,这样我们要计算的函数就是\textbf{完全积性函数}了。

考虑埃氏筛的过程,设 $g_{n,k}$ 表示\textbf{筛了 $2 \sim n$ 中的数,用前 $k$ 个质数来筛,没有被筛掉的数的 $f'$ 值之和}。

设 $p_i$ 表示第 $i$ 个质数,考虑求出 $g$ 数组,首先容易发现如果 $p_k^2 > n$,$g_{n,k}=g_{n,k-1}$。

否则我们会筛掉一些数,由于是完全积性函数,容易得出转移:

$$g_{n,k}=g_{n,k-1}-f'(p_k) \times (g_{\lfloor \frac{n}{p_k} \rfloor,k-1}-\sum_{i=1}^{k-1}f'(p_i))$$

注意 $g$ 数组计算的答案是\textbf{“将没被筛掉的数也当成质数”时的答案}。

我们考虑计算答案,设 $s_{n,k}$ 表示\textbf{筛了 $2 \sim n$ 中的数,用前 $k$ 个质数来筛,没有被筛掉的数的 $f$ 值之和。}

现在这个 $s_{n,k}$ 求的就是真正的答案了,最终答案就是 $s_{n,0}+f(1)$。

设 $sp_i=\sum_{j=1}^i p_j$,$D$ 为最大的 $i$ 满足 $p_i^2 \le n$,容易得出转移:

$$s_{n,k}=g_{n,D}-sp_{k}+\sum_{p_i^e \le n \& i>k} f({p_i^e})(s_{\frac{n}{p_i^e},e}+[e>1])$$

然后暴力递归求出 $s$ 就行了,不需要记忆化。

\subsubsection{求素数个数}

\begin{minted}{c++}
// == Preparations ==
#define int long long
#define id(x) ((x<=lim)?(id1[x]):(id2[n/(x)]))
#define rep(i,j,k) for(int i=j;i<=k;++i)
#define per(i,j,k) for(int i=j;i>=k;--i)
int const N=1e6+10;
int lim,prime[N],w[N],g[N],id1[N],id2[N],m;bool vis[N];
// == Main ==
inline void init(){
    //预处理根号以内质数
    rep(i,2,lim){
        if (!vis[i]) prime[++prime[0]]=i;
        for (int j=1;j<=prime[0] && i*prime[j]<=lim;++j){
            vis[i*prime[j]]=1;
            if (i%prime[j]==0) break;
        }
    }
}
inline void solve(){
    int n;cin>>n,lim=sqrtl(n),init();
    for (int l=1,r;l<=n;l=r+1){
        r=n/(n/l),w[++m]=n/l,g[m]=w[m]-1;
        if (w[m]<=lim) id1[w[m]]=m;
        else id2[n/w[m]]=m;
        //预处理 g 数组第一维有可能的取值
    }
    rep(i,1,prime[0])
        for (int j=1;j<=m && prime[i]*prime[i]<=w[j];++j)
            g[j]-=g[id(w[j]/prime[i])]-(i-1);
    cout<<g[id(n)]<<'\n';
}
\end{minted}

\subsubsection{完整模板}

\begin{minted}{c++}
// == Preparations ==
#define int long long
#define rep(i,j,k) for(int i=j;i<=k;++i)
#define per(i,j,k) for(int i=j;i>=k;--i)
#define id(x) ((x<=lim)?(id1[x]):(id2[n/(x)]))
#define add(x,y) (x=((x+y>=mod)?(x+y-mod):(x+y)))
// == Main ==
//例:求 f(p^k)=p^k(p^k-1) 的前缀和
//发现 f(p^k)=p^{2k}-p^k
int const mod=1e9+7;
int n;
inline int f(int op,int x){
    x%=mod;
    if (!op) return x;
    return x*x%mod;
}
inline int F(int x){
    x%=mod;
    return x*(x-1)%mod;
}
inline int smf(int op,int x){
    x%=mod;
    if (!op) return (x*(x+1)/2)%mod;
    return x*(x+1)%mod*(2*x%mod+1)%mod*166666668%mod;
}
//smf -> f 前缀和
int const N=1e6+10;
int lim,prime[N],sp1[N],sp2[N],w[N],g1[N],g2[N],id1[N],id2[N],m;bool vis[N];
inline void init(){
    //预处理根号以内质数
    rep(i,2,lim){
        if (!vis[i]) prime[++prime[0]]=i;
        for (int j=1;j<=prime[0] && i*prime[j]<=lim;++j){
            vis[i*prime[j]]=1;
            if (i%prime[j]==0) break;
        }
    }
    rep(i,1,prime[0]){
        sp1[i]=sp1[i-1],add(sp1[i],f(0,prime[i]));
        sp2[i]=sp2[i-1],add(sp2[i],f(1,prime[i]));
    }
}
inline int S(int x,int y){
    if (prime[y]>=x) return 0;
    int ans=((g2[id(x)]+mod-g1[id(x)])%mod+mod-(sp2[y]-sp1[y]+mod)%mod)%mod;
    for (int i=y+1;i<=prime[0] && prime[i]*prime[i]<=x;++i)
        for (int e=1,gg=prime[i];gg<=x;++e,gg*=prime[i])
            add(ans,F(gg)*(S(x/gg,i)+(e>1))%mod);
    return ans;
}
inline void solve(){
    cin>>n,lim=sqrtl(n),init();
    for (int l=1,r;l<=n;l=r+1){
        r=n/(n/l),w[++m]=n/l;
        g1[m]=(smf(0,w[m])+mod-1)%mod;
        g2[m]=(smf(1,w[m])+mod-1)%mod;
        if (w[m]<=lim) id1[w[m]]=m;
        else id2[n/w[m]]=m;
        //预处理 g 数组第一维有可能的取值
    }
    rep(i,1,prime[0])
        for (int j=1;j<=m && prime[i]*prime[i]<=w[j];++j)
            add(g1[j],mod-f(0,prime[i])*(g1[id(w[j]/prime[i])]+mod-sp1[i-1])%mod),
            add(g2[j],mod-f(1,prime[i])*(g2[id(w[j]/prime[i])]+mod-sp2[i-1])%mod);
    cout<<(S(n,0)+1)%mod<<'\n';
}
\end{minted}
