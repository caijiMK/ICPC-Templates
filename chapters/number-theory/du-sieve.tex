\section{杜教筛}

\subsection{狄利克雷卷积}

定义两个数论函数 $f$ 和 $g$ 的狄利克雷卷积为:$(f*g)(n)=\sum_{d|n}f(d) \times g(\frac{n}{d})$。

一些常见的卷积式:

$$\mu*I=\epsilon$$

$$\phi*I=id$$

\subsection{杜教筛}

$f$ 是一个积性函数,我们需要求 $f$ 的前缀和。

考虑构造两个积性函数 $h,g$,满足 $h=f*g$。

设 $S(n)=\sum_{i=1}^n f(i)$。

套路式:

$$\sum_{i=1}^n h(i)=\sum_{i=1}^n\sum_{d|i}g(d) \times f(\frac{i}{d})$$

$$\sum_{i=1}^n h(i)=\sum_{i=1}^n\sum_{d=1}^{\lfloor\frac{n}{i}\rfloor} g(i) \times f(d)$$

$$\sum_{i=1}^n h(i)=\sum_{i=1}^n g(i) \times S(\lfloor \frac{n}{i} \rfloor)$$

然后将右边式子的第一项提出来,可以得到:

$$\sum_{i=1}^n h(i)=g(1) \times S(n)+\sum_{i=2}^n g(i) \times S(\lfloor \frac{n}{i} \rfloor)$$

$$g(1) \times S(n)=\sum_{i=1}^n h(i)-\sum_{i=2}^n g(i) \times S(\lfloor \frac{n}{i} \rfloor)$$

经过复杂度分析后得出:如果能快速求出 $h$ 和 $g$ 的前缀和,求 $S(n)$ 的过程递归处理即可得到 $\Theta(n^{\frac{2}{3}})$ 的复杂度,\textbf{需要记忆化,并且先预处理出一部分值}。

\subsection{示例代码}

求 $\phi/\mu$ 的前缀和:

\begin{minted}{c++}
//A tree without skin will surely die.
//A man without face is invincible.
#include<bits/stdc++.h>
using namespace std;
#define int long long
int const N=1e7+10;
int const maxn=1e7;
int phi[N],prime[N],mu[N];
bitset<N>vis;
inline void init(){
    phi[1]=mu[1]=1;
    for (int i=2;i<=maxn;++i){
        if (!vis[i])
            prime[++prime[0]]=i,phi[i]=i-1,mu[i]=-1;
        for (int j=1;j<=prime[0] && prime[j]*i<=maxn;++j){
            vis[prime[j]*i]=1;
            if (i%prime[j]==0){
                phi[i*prime[j]]=phi[i]*prime[j];
                break;
            }
            phi[i*prime[j]]=phi[i]*phi[prime[j]];
            mu[i*prime[j]]=-mu[i];
        }
    }
    for (int i=2;i<=maxn;++i)
        phi[i]+=phi[i-1],mu[i]+=mu[i-1];
}
//mu*I=e
//f *g=h
unordered_map<int,int>mpmu,mpphi;
inline int summu(int n){
    if (n<=maxn) return mu[n];
    if (mpmu[n]) return mpmu[n];
    int ans=1;
    for (int l=2,r;l<=n;l=r+1)
        r=n/(n/l),ans-=(r-l+1)*summu(n/l);
    return mpmu[n]=ans;
}
//phi*I=id
//f  *g=h
inline int sumphi(int n){
    if (n<=maxn) return phi[n];
    if (mpphi[n]) return mpphi[n];
    int ans=n*(n+1)/2;
    for (int l=2,r;l<=n;l=r+1)
        r=n/(n/l),ans-=(r-l+1)*sumphi(n/l);
    return mpphi[n]=ans;
}
void solve(){
    int n;cin>>n;
    cout<<sumphi(n)<<' '<<summu(n)<<'\n';
}
signed main(){
    ios::sync_with_stdio(false);
    cin.tie(0),cout.tie(0);
    int t=1;
    cin>>t,init();
    while (t--) solve();
    return 0;
}
\end{minted}