\section{min25 筛}

\subsection{简述}

应用范围:求 $\sum_{i=1}^n f(i)$,其中 $f(i)$ 是积性函数。

需要满足 $f(i)$ 在 $i$ 是质数时的取值是多项式。

时间复杂度:$\Theta(n^{1-\epsilon})$/$\Theta(\frac{n^{\frac{3}{4}}}{\log n})$。

主要想法是将 $f(i)$ 分成三个部分后求和:$i$ 是质数,$i$ 是合数,$i=1$。

先求出 $i$ 是质数部分的和。

为了方便起见,我们将 $f(i)$ 分解成若干 $x^k$ 之和,然后分别计算,这样我们要计算的函数就是\textbf{完全积性函数}了。

考虑埃氏筛的过程,设 $g_{n,k}$ 表示\textbf{筛了 $2 \sim n$ 中的数,用前 $k$ 个质数来筛,没有被筛掉的数的 $f'$ 值之和}。

设 $p_i$ 表示第 $i$ 个质数,考虑求出 $g$ 数组,首先容易发现如果 $p_k^2 > n$,$g_{n,k}=g_{n,k-1}$。

否则我们会筛掉一些数,由于是完全积性函数,容易得出转移:

$$g_{n,k}=g_{n,k-1}-f'(p_k) \times (g_{\lfloor \frac{n}{p_k} \rfloor,k-1}-\sum_{i=1}^{k-1}f'(p_i))$$

注意 $g$ 数组计算的答案是\textbf{“将没被筛掉的数也当成质数”时的答案}。

我们考虑计算答案,设 $s_{n,k}$ 表示\textbf{筛了 $2 \sim n$ 中的数,用前 $k$ 个质数来筛,没有被筛掉的数的 $f$ 值之和。}

现在这个 $s_{n,k}$ 求的就是真正的答案了,最终答案就是 $s_{n,0}+f(1)$。

设 $sp_i=\sum_{j=1}^i p_j$,$D$ 为最大的 $i$ 满足 $p_i^2 \le n$,容易得出转移:

$$s_{n,k}=g_{n,D}-sp_{k}+\sum_{p_i^e \le n \& i>k} f({p_i^e})(s_{\frac{n}{p_i^e},e}+[e>1])$$

然后暴力递归求出 $s$ 就行了,不需要记忆化。

\subsection{求素数个数}

\begin{minted}{c++}
// == Preparations ==
#define int long long
#define id(x) ((x<=lim)?(id1[x]):(id2[n/(x)]))
#define rep(i,j,k) for(int i=j;i<=k;++i)
#define per(i,j,k) for(int i=j;i>=k;--i)
int const N=1e6+10;
int lim,prime[N],w[N],g[N],id1[N],id2[N],m;bool vis[N];
// == Main ==
inline void init(){
    //预处理根号以内质数
    rep(i,2,lim){
        if (!vis[i]) prime[++prime[0]]=i;
        for (int j=1;j<=prime[0] && i*prime[j]<=lim;++j){
            vis[i*prime[j]]=1;
            if (i%prime[j]==0) break;
        }
    }
}
inline void solve(){
    int n;cin>>n,lim=sqrtl(n),init();
    for (int l=1,r;l<=n;l=r+1){
        r=n/(n/l),w[++m]=n/l,g[m]=w[m]-1;
        if (w[m]<=lim) id1[w[m]]=m;
        else id2[n/w[m]]=m;
        //预处理 g 数组第一维有可能的取值
    }
    rep(i,1,prime[0])
        for (int j=1;j<=m && prime[i]*prime[i]<=w[j];++j)
            g[j]-=g[id(w[j]/prime[i])]-(i-1);
    cout<<g[id(n)]<<'\n';
}
\end{minted}

\subsection{完整模板}

\begin{minted}{c++}
// == Preparations ==
#define int long long
#define rep(i,j,k) for(int i=j;i<=k;++i)
#define per(i,j,k) for(int i=j;i>=k;--i)
#define id(x) ((x<=lim)?(id1[x]):(id2[n/(x)]))
#define add(x,y) (x=((x+y>=mod)?(x+y-mod):(x+y)))
// == Main ==
//例:求 f(p^k)=p^k(p^k-1) 的前缀和
//发现 f(p^k)=p^{2k}-p^k
int const mod=1e9+7;
int n;
inline int f(int op,int x){
    x%=mod;
    if (!op) return x;
    return x*x%mod;
}
inline int F(int x){
    x%=mod;
    return x*(x-1)%mod;
}
inline int smf(int op,int x){
    x%=mod;
    if (!op) return (x*(x+1)/2)%mod;
    return x*(x+1)%mod*(2*x%mod+1)%mod*166666668%mod;
}
//smf -> f 前缀和
int const N=1e6+10;
int lim,prime[N],sp1[N],sp2[N],w[N],g1[N],g2[N],id1[N],id2[N],m;bool vis[N];
inline void init(){
    //预处理根号以内质数
    rep(i,2,lim){
        if (!vis[i]) prime[++prime[0]]=i;
        for (int j=1;j<=prime[0] && i*prime[j]<=lim;++j){
            vis[i*prime[j]]=1;
            if (i%prime[j]==0) break;
        }
    }
    rep(i,1,prime[0]){
        sp1[i]=sp1[i-1],add(sp1[i],f(0,prime[i]));
        sp2[i]=sp2[i-1],add(sp2[i],f(1,prime[i]));
    }
}
inline int S(int x,int y){
    if (prime[y]>=x) return 0;
    int ans=((g2[id(x)]+mod-g1[id(x)])%mod+mod-(sp2[y]-sp1[y]+mod)%mod)%mod;
    for (int i=y+1;i<=prime[0] && prime[i]*prime[i]<=x;++i)
        for (int e=1,gg=prime[i];gg<=x;++e,gg*=prime[i])
            add(ans,F(gg)*(S(x/gg,i)+(e>1))%mod);
    return ans;
}
inline void solve(){
    cin>>n,lim=sqrtl(n),init();
    for (int l=1,r;l<=n;l=r+1){
        r=n/(n/l),w[++m]=n/l;
        g1[m]=(smf(0,w[m])+mod-1)%mod;
        g2[m]=(smf(1,w[m])+mod-1)%mod;
        if (w[m]<=lim) id1[w[m]]=m;
        else id2[n/w[m]]=m;
        //预处理 g 数组第一维有可能的取值
    }
    rep(i,1,prime[0])
        for (int j=1;j<=m && prime[i]*prime[i]<=w[j];++j)
            add(g1[j],mod-f(0,prime[i])*(g1[id(w[j]/prime[i])]+mod-sp1[i-1])%mod),
            add(g2[j],mod-f(1,prime[i])*(g2[id(w[j]/prime[i])]+mod-sp2[i-1])%mod);
    cout<<(S(n,0)+1)%mod<<'\n';
}
\end{minted}
