\section{剩余}

\begin{tcolorbox}
\textbf{定义}:

令整数 $k\geq 2$ ,整数 $a$ , $m$ 满足 $(a,m)=1$ ,若存在整数 $x$ 使得
\begin{equation}\tag{1}
x^k\equiv a\pmod m 
\end{equation}

则称 $a$ 为模 $m$ 的 $k$ 次剩余,否则称 $a$ 为模 $m$ 的 $k$ 次非剩余。
\end{tcolorbox}

当整数 $k\geq 2$ ,整数 $a$ , $m$ 满足 $(a,m)=1$ ,模 $m$ 有原根 $g$ 时,令 $d=(k,\varphi(m))$ ,则:

\begin{tcolorbox}
\textbf{性质 1}:$a$ 为模 $m$ 的 $k$ 次剩余当且仅当 $d\mid \operatorname{ind}_g a$ ,即:$ a^{\frac{\varphi(m)}{d}}\equiv 1\pmod m $
\end{tcolorbox}

\begin{tcolorbox}
\textbf{证明 1}:令 $x=g^y$ ,则方程 $(1)$ 等价于
$ g^{ky}\equiv g^{\operatorname{ind}_g a}\pmod m $

其等价于
$ ky\equiv \operatorname{ind}_g a\pmod{\varphi(m)} $

由同余的性质,我们知道 $y$ 有整数解当且仅当 $d=(k,\varphi(m))\mid \operatorname{ind}_g a$ ,进而
\begin{align} 
\notag a^{\frac{\varphi(m)}{d}}\equiv 1\pmod m
&\notag \iff g^{\frac{\varphi(m)}{d}\operatorname{ind}_g a}\equiv 1\pmod m\\ 
&\notag \iff \varphi(m)\mid\frac{\varphi(m)}{d}\operatorname{ind}_g a\\ 
&\notag \iff \varphi(m)d\mid \varphi(m)\operatorname{ind}_g a\\ 
&\notag \iff d\mid \operatorname{ind}_g a
\end{align}

\end{tcolorbox}

\begin{tcolorbox}
\textbf{性质 2}:方程 $(1)$ 若有解,则模 $m$ 下恰有 $d$ 个解
\end{tcolorbox}

\begin{tcolorbox}
\textbf{性质 3}:模 $m$ 的 $k$ 次剩余类的个数为 $\dfrac{\varphi(m)}{d}$ , 其有形式
    $$ a\equiv g^{di}\pmod m,\qquad \left(0\leq i<\frac{\varphi(m)}{d}\right) $$
\end{tcolorbox}

\subsection{二次剩余}

下面讨论模数是奇素数时的二次剩余问题:

$$x^2 \equiv n \pmod p$$

参考以上结论不难得出,也就说二次剩余的数量恰为 $\frac{p-1}{2}$ ,其他的非 $0$ 数都是非二次剩余,数量也是 $\frac{p-1}{2}$;判别是否为二次剩余只需检查是否 $n^{\frac{p-1}{2}} \equiv 1 \pmod p$ 即可。

\subsection{Cipolla}

通过随机找到一个 $a$ 满足 $a^2 - n$ 是非二次剩余。

定义 $i^2\equiv a^2 - n$,将所有数表示为 $A+Bi$ 的形式,有 $(a+i)^{p+1}\equiv n$。

那么 $(a+i)^{\frac{p-1}{2}}$ 即是一个解,其相反数是另一个解。

\begin{minted}{c++}
// == Preparations ==
typedef long long ll;
int mod , w2;
mt19937 rnd(114514);
struct Complex
{
    int a , b;
    Complex(ll aa = 0 , ll bb = 0):a(aa % mod) , b(bb % mod){}
};
typedef Complex cp;
cp operator + (cp x , cp y){return cp(x.a + y.a , x.b + y.b);}
cp operator - (cp x , cp y){return cp(x.a - y.a , x.b - y.b);}
cp operator * (cp x , cp y)
{
    return cp((ll)x.a * y.a + (ll)w2 * x.b % mod * y.b ,
              (ll)x.b * y.a + (ll)y.b * x.a);
}
cp Qpow(cp x , int p)
{
    cp res = 1;
    while(p)
    {
        if(p & 1)res = res * x;
        x = x * x;
        p >>= 1;
    }
    return res;
}
// == Main ==
int Legendre(int a){return Qpow(a , (mod - 1) / 2).a;}
pair<int , int> Cipolla(int a)
{
    if(a % mod == 0)return {0 , -1};
    if(Legendre(a) == mod - 1)return {-1 , -1};
    int res = 0;
    while(1)
    {
        int r = rnd() % mod;
        w2 = ((ll)r * r - a) % mod;
        if(w2 < 0)w2 += mod;
        if(w2 == 0)
        {
            res = r;
            break ;
        }
        if(Legendre(w2) == mod - 1)
        {
            res = Qpow(cp(r , 1) , (mod + 1) / 2).a;
            break ;
        }
    }
    if(res * 2 >= mod)res = mod - res;
    return {res , mod - res};
}
\end{minted}

\subsection{模素数 N 次剩余}

我们令 $x=g^c$ , $g$ 是 $p$ 的原根(我们一定可以找到这个 $g$ 和 $c$ ),问题转化为求解 $(g^c)^a \equiv b \pmod p$ . 稍加变换,得到
$$ (g^a)^c \equiv b \pmod p $$

于是就转换成了 BSGS 的基本模型了,可以在 $O(\sqrt p)$ 解出 $c$ ,这样可以得到原方程的一个特解 $x_0\equiv g^c\pmod p$ .

