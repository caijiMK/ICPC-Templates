\section{Border 理论}

\begin{tcolorbox}
\textbf{定理 1}:对于一个字符串 $s$,若用 $t$ 表示其最长的 Border,则有 $\mathcal{B}(s) = \mathcal{B}(t) \cup \{t\}$。
\end{tcolorbox}

\begin{tcolorbox}
\textbf{定理 2}:一个字符串的 Border 与 Period 一一对应。具体地,$\mathrm{pre}(s, i) \in \mathcal{B}(s) \iff |s| - i \in \mathcal{P}(s)$。
\end{tcolorbox}

\begin{tcolorbox}
\textbf{弱周期引理}:

$$
\forall p, q\in\mathcal{P}(s), p + q \le |s| \implies \gcd(p, q)\in\mathcal{P}(s)
$$
\end{tcolorbox}

\begin{tcolorbox}
\textbf{定理 3}:若字符串 $t$ 是字符串 $s$ 的前缀,且 $a \in \mathcal{P}(s), b \in \mathcal{P}(t), b \mid a, |t| \ge a$,且 $b$ 是 $t$ 的整周期,则有 $b \in \mathcal{P}(s)$。
\end{tcolorbox}

\begin{tcolorbox}
\textbf{周期引理}:

$$
\forall p, q\in\mathcal{P}(s), p + q - \gcd(p, q) \le |s| \implies \gcd(p, q)\in\mathcal{P}(s)
$$
\end{tcolorbox}

\begin{tcolorbox}
\textbf{定理 4}:对于文本串 $s$ 和模式串 $t$,若 $|t| \ge \frac{|s|}{2}$,且 $t$ 在 $s$ 中至少成功匹配了 $3$ 次,则每次匹配的位置形成一个等差数列,且公差为 $t$ 的最小周期。
\end{tcolorbox}

\begin{tcolorbox}
\textbf{定理 5}:一个字符串 $s$ 的所有长度不小于 $\frac{|s|}{2}$ 的 Border 的长度构成一个等差数列。
\end{tcolorbox}

\begin{tcolorbox}
\textbf{定理 6}:一个字符串的所有 Border 的长度排序后可以划分成 $\lceil\log_2|s|\rceil$ 个连续段,使得每段都是一个等差数列。
\end{tcolorbox}

\begin{tcolorbox}
\textbf{定理 7}:回文串的回文前/后缀即为该串的 Border。
\end{tcolorbox}

\begin{tcolorbox}
\textbf{定理 8}:若回文串 $s$ 有周期 $p$,则可以把 $\mathrm{pre}(s, p)$ 划分成长度为 $|s| \bmod p$ 的前缀和长度为 $p - |s| \bmod p$ 的后缀,使得它们都是回文串。
\end{tcolorbox}

\begin{tcolorbox}
\textbf{定理 9}:若 $t$ 是回文串 $s$ 的最长 Border 且 $|t| \ge \frac{|s|}{2}$,则 $t$ 在 $s$ 中只能匹配 $2$ 次。
\end{tcolorbox}

\begin{tcolorbox}
\textbf{定理 10}:对于任意一个字符串以及 $u, v\in\mathrm{Ssuf}(s), |u| < |v|$,一定有 $u$ 是 $v$ 的 Border。
\end{tcolorbox}

\begin{tcolorbox}
\textbf{定理 11}:对于任意一个字符串 $s$ 以及 $u, v\in\mathrm{Ssuf}(s), |u| < |v|$,一定有 $2|u| \le |v|$。
\end{tcolorbox}

\begin{tcolorbox}
\textbf{定理 12}:$|\mathrm{Ssuf}(s)| \le \log_2|s|$。
\end{tcolorbox}