\section{K-D Tree}

\begin{minted}{c++}
// == Preparations ==
#include <array>
#include <vector>
// == Main ==
template<const int Dim = 2>
struct KDTree {
    using point = array<int, Dim>;
    struct node {
        point p, l, r;
        int val, siz, sum;
        node *ls, *rs;

        node() = default;
        node(point _p, int _val = 0):
            p(_p), l(_p), r(_p), val(_val), siz(1), sum(_val), ls(nullptr), rs(nullptr) {}
        void pushup() {
            l = r = p, siz = 1, sum = val;
            for (int i = 0; i < Dim; i++) {
                if (ls) l[i] = min(l[i], ls->l[i]), r[i] = max(r[i], ls->r[i]);
                if (rs) l[i] = min(l[i], rs->l[i]), r[i] = max(r[i], rs->r[i]);
            }
            if (ls) siz += ls->siz, sum += ls->sum;
            if (rs) siz += rs->siz, sum += rs->sum;
            return ;
        }
    };
    vector<node *> root;
    using itor = typename vector<node>::iterator;

    node *build(itor l, itor r, int dim = 0) {
        if (l == r) return nullptr;
        int mid = (r - l) / 2;
        nth_element(l, l + mid, r, [&dim](const node &x, const node &y) {return x.p[dim] < y.p[dim];});
        node *now = new node(*(l + mid));
        now->ls = build(l, l + mid, (dim + 1) % Dim);
        now->rs = build(l + mid + 1, r, (dim + 1) % Dim);
        now->pushup();
        return now;
    }
    void getnode(node *now, vector<node> &vec) {
        if (!now) return ;
        vec.push_back(*now);
        getnode(now->ls, vec), getnode(now->rs, vec);
        delete now;
        return ;
    }
    void insert(point p, int val) {
        vector<node> tmp({node(p, val)});
        while (!root.empty() && root.back()->siz == (int)tmp.size())
            getnode(root.back(), tmp), root.pop_back();
        sort(tmp.begin(), tmp.end(), [](const node &x, const node &y) {return x.p < y.p;});
        vector<node> vec;
        for (node i : tmp)
            if (!vec.empty() && vec.back().p == i.p) vec.back().val += i.val;
            else vec.push_back(i);
        root.push_back(build(vec.begin(), vec.end()));
        return ;
    }
    int query(point ll, point rr, node *now) {
        if (!now) return 0;
        int flag = 1;
        for (int i = 0; i < Dim; i++)
            if (now->r[i] < ll[i] || now->l[i] > rr[i]) return 0;
            else flag &= ll[i] <= now->l[i] && now->r[i] <= rr[i];
        if (flag) return now->sum;
        flag = 1;
        for (int i = 0; i < Dim; i++) flag &= ll[i] <= now->p[i] && now->p[i] <= rr[i];
        return flag * now->val + query(ll, rr, now->ls) + query(ll, rr, now->rs);
    }
    int query(point ll, point rr) {
        int ans = 0;
        for (node *rt : root) ans += query(ll, rr, rt);
        return ans;
    }
    ~KDTree() {
        vector<node> tmp;
        for (node *rt : root) getnode(rt, tmp);
    }
};
\end{minted}

最近最远点查询:每次走到一个子树时判断到这个矩形的最近/最远点是否劣于答案,如果是则直接退出。

\textbf{复杂度不正确!}

\begin{minted}{c++}
// == Main ==
// 代码为查询 k 第远的点,q 为小根堆。
void query(point p, node *now) {
    if (!now) return;
    long long mxdis = 0, dis = 0;
    for (int i = 0; i < Dim; i++) {
        mxdis += max((long long)(now->l[i] - p[i]) * (now->l[i] - p[i]),
            (long long)(now->r[i] - p[i]) * (now->r[i] - p[i]));
        dis += (long long)(now->p[i] - p[i]) * (now->p[i] - p[i]);
    }
    q.push(dis);
    if ((int)q.size() > k) q.pop();
    if ((int)q.size() == k && mxdis < q.top()) return;
    query(p, now->ls), query(p, now->rs);
    return;
}
\end{minted}