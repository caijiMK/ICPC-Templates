\section{群论(Burnside \& Polya)}

\subsection{定义}

定义集合 $G$ 和作用与集合 $G$ 的二元运算 $\times$,若其满足:封闭性,结合律,存在单位元,存在逆元则被称为群。

若 $H$ 为 $G$ 的一个子集,且 $(H,\times)$ 构成一个群,则称 $(H,\times)$ 是 $(G,\times)$ 的一个子群。

定义左陪集:$g \in G$,若满足 $gH=g \times h,h \in H$,则称其为 $H$ 在 $G$ 内的关于 $g$ 的左陪集。

右陪集也类似定义。

根据这些定义有一些性质:

\begin{itemize}
    \item $H$ 的全体右陪集的并是 $G$。
    \item 两个 $H$ 的右陪集要么交集是空要么相等。
\end{itemize}

同时定义 $[G:H]$ 表示 $G$ 中 $H$ 的不同右陪集数量。

\subsection{拉格朗日定理}

对于有限群 $(G,\times)$ 和有限群 $(H,\times)$,若 $H$ 为 $G$ 的子群,则满足 $|G|$ 是 $|H|$ 的倍数。

也可以写成:$|H| \times |[G:H]|=|G|$。

\subsection{置换}

定义 $\sigma=(a_1,a_2,\dots,a_n)$ 表示一个置换。

定义置换的运算(也称乘法) $\sigma(a)=(a_{\sigma_1},a_{\sigma_2},\dots,a_{\sigma_n})$。

\subsection{置换群}

令集合 $N=\{1,2,\dots,n\}$,令集合 $M$ 为 $N$ 的若干排列构成的集合,如果 $M$ 关于置换的乘法满足群的性质,我们称群 $G=(M,\times)$ 是一个置换群。

\subsection{群作用}

分为左群作用和右群作用。下面描述的是左群作用的定义,下文出于方便,将同一称为「群作用」,并使用此处的定义。

定义:

对于一个集合 $M$ 和群 $G$。

若给定了一个二元函数 $\phi(v,k)$ 其中 $v$ 为群中元素,$k$ 为集合中元素,$e$ 是单位元,且有:

$$\phi(e,k)=k$$

$$\phi(g,\phi(s,k))=\phi(g \times s,k)$$

则称群 $G$ 作用于集合 $M$。

\subsection{轨道-稳定子定理}

考虑一个作用在 $X$ 上的群 $G$。$X$ 中一个元素 $x$ 的「轨道」则是 $x$ 通过 $G$ 中的元素可以转移到的元素的集合。$x$ 的轨道被记为 $G(x)$,方便起见,我们设 $g(x)=\phi(g,x)$。

稳定子被定义为:$G^x=\{g|g\in G,g(x)=x\}$。

给一个例子:

有若干个 $2 \times 2$,里面元素是黑或白的矩阵,设这些矩阵构成的集合是 $M$,给定一个群 $G$,$G$ 中元素是:顺时针旋转 $90°/180°/270°/360°$。

那么对于一个 $M$ 中的元素 $x=\left(\begin{array}{cc}
    1 & 1 \\
    0 & 0
    \end{array}\right)$,其稳定子 $G^x$ 为:顺时针旋转 $0°$。

其轨道为:$\left(\begin{array}{cc}
    1 & 1 \\
    0 & 0
    \end{array}\right),\left(\begin{array}{cc}
        0 & 1 \\
        0 & 1
        \end{array}\right),\left(\begin{array}{cc}
            0 & 0 \\
            1 & 1
            \end{array}\right),\left(\begin{array}{cc}
                1 & 0 \\
                1 & 0
                \end{array}\right)$

同时,根据这个例子我们也可以发现轨道大小乘稳定子大小刚好是群的大小!

这个东西叫做“轨道-稳定子定理”:$|G^x| \times |G(x)|=|G|$。

\subsection{Burnside 定理}

公式:定义 $G$ 为一个置换群,定义其作用于 $X$,如果 $x,y \in X$ 在 $G$ 作用下可以相等(存在 $f \in G$ 使得 $f(x)=y$)则定义 $x,y$ 属于一个等价类,不同的等价类数量即为:

$$|X/G|=\frac{1}{|G|}\sum_{g \in G}X^g$$

人话:$X$ 在群 $G$ 作用下的等价类总数等于每一个 $g$ 作用于 $X$ 的不动点的算数平均值。

例题:

给定一个 $n$ 个点,$n$ 条边的环,有 $n$ 种颜色,给每个顶点染色,问有多少种本质不同的染色方案,答案对 $10^9+7$ 取模。

注意本题的本质不同,定义为:只需要不能通过旋转与别的染色方案相同。

我们设群 $G=\{$旋转 $0$ 个,$\dots$,旋转 $n-1$ 个$\}$,同时设 $X$ 为把环当成序列所有染色方案的集合。

那么 $ans=\frac{1}{|G|} \sum_{g \in G} X^g=\frac{1}{n} \sum_{i=1}^n n^{\gcd(i,n)}$。

\subsection{Polya 定理}

发现我们需要求出的只是不动点数量,经过分析后可以发现对于某个置换 $g$,$g$ 的某个置换环内所有点应当染成相同的颜色。

因此设 $m$ 为颜色数,$c(g)$ 为置换 $g$ 的环数,可以得到:$ans=\frac{1}{|G|} \sum_{g \in G} m^{c(g)}$。