\section{最大团搜索}

给定一张无向图,Bron-Kerbosch 算法可以搜索出所有极大团。其时间复杂度与极大团数量相同,为 $\mathcal{O}(3^{\frac{n}{3}})$。

我们定义一张图的 degeneracy 为最小的 $d$ 满足其任意导出子图都存在至少一个点度数 $\le d$,则复杂度为 $\mathcal{O}(dn3^{\frac{d}{3}})$。

\begin{minted}{c++}
// == Preparations ==
#define popc __builtin_popcountll 
typedef long long ll;
const int N = 55;
int n , ans , a[N]; ll g[N] , d[N]; //g is graph matrix , b[0...n-1][0...n-1]
// == Main ==
void Dfs(ll R , ll P , ll X)
{
    if(!P)
    {
        if(!X)ans = max(ans , popc(R));//now R is the maximal clique
        return ;
    }
    ll v = __lg(P | X);
    for(ll s = P ^ (P & g[v]) ; s ; s -= s & -s)
    {
        ll u = __lg(s & -s);
        Dfs(R | (1ll << u) , P & g[u] , X & g[u]);
        P ^= 1ll << u , X |= 1ll << u;
    }
}
void BronKerbosch()
{
    for(int i = 0 ; i < n ; i++)d[i] = g[i];
    for(int i = 0 ; i < n ; i++)
    {
        int &k = a[i]; k = -1;
        for(int j = 0 ; j < n ; j++)
            if(d[j] != -1 && (k == -1 || popc(d[j]) <= popc(d[k])))
                k = j;
        d[k] = -1;
        for(int j = 0 ; j < n ; j++)
            if(d[j] >= 0 && (d[j] >> k) & 1)d[j] ^= 1ll << k;
    }
    ll P = (1ll << n) - 1 , X = 0;
    for(int i = 0 ; i < n ; i++)
    {
        int u = a[i];
        Dfs(1ll << u , P & g[u] , X & g[u]);
        P ^= 1ll << u , X |= 1ll << u;
    }
}
\end{minted}