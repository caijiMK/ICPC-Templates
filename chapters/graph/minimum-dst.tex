\section{最小树形图}

\subsection{朱刘算法}

基本想法:对除根节点外的所有点,选一条边权最小的入边;对于入边形成的简单环,最多只有一条边会被删除,将环缩成一个点,并改变与其相连边的边权。

如果有环,每轮至少减少一个点。时间复杂度 $\mathcal{O}(nm)$。

\begin{minted}{c++}
bool ZhuLiu(int n , int m , int rt)
{
    while(1)
    {
        fill(mn + 1 , mn + n + 1 , 1e9);
        fill(id + 1 , id + n + 1 , 0);
        fill(top + 1 , top + n + 1 , 0);
        for(int i = 1 ; i <= m ; i++)
        {
            auto [u , v , w] = e[i];
            if(w < mn[v])mn[v] = w , fa[v] = u;
        }
        for(int i = 1 ; i <= n ; i++)
        {
            if(i == rt)continue ;
            if(mn[i] != 1e9)ans += mn[i];
            else return 0;
        }
        int cnt = 0 , tot = 0;
        for(int i = 1 ; i <= n ; i++)
        {
            if(id[i] || i == rt)continue ;
            int u = i;
            for(; u != rt && !top[u] ; u = fa[u])top[u] = i;
            if(top[u] == i)
            {
                id[u] = ++cnt;
                for(int v = fa[u] ; v != u ; v = fa[v])
                    id[v] = cnt;
            }
        }
        if(!cnt)return 1;
        for(int i = 1 ; i <= n ; i++)
            if(!id[i])id[i] = ++cnt;
        for(int i = 1 ; i <= m ; i++)
        {
            auto [u , v , w] = e[i];
            if(id[u] == id[v])continue ;
            e[++tot] = {id[u] , id[v] , w - mn[v]};
        }
        n = cnt , m = tot , rt = id[rt];
    }
}
\end{minted}

\subsection{可并堆优化}

上述算法的瓶颈在于对每个点(包括缩环得到的点)找边权最小的入边。

考虑用可并堆存储每个点 $u$ 的入边,为对该点入边边权同时减去 $w(\text{in}(u))$,堆需要支持全局减。

为了避免每次对没有缩环的点的遍历,将过程改为迭代式的。

同时维护两个并查集,分别记录连通块信息、当前所在环的编号。

为保证图能被缩成一个点,需要先加入边 $(i,i \bmod n+1,+\infty)$。

\begin{tcolorbox}
\textbf{contraction 树}:为了计算方案,引入一种树结构,树上的叶子是图的原始节点,每个非叶子节点都表示一个环,每个点的父亲是它所在的环。
\end{tcolorbox}

收缩过程中,对每个环内的点记录其所在环的编号 $f$、环上的前驱(即树上的兄弟)$pre$、以及入边 $\text{in}$,即可刻画出该结构。

由于优化后的缩环过程与 $r$ 无关,对于任意合法的根 $r$,都可以通过展开该结构得到以其为根的最小树形图。展开过程:

\begin{enumerate}

    \item 对于任意合法的根节点 $r$,从 $r$ 开始跳父亲直到 contraction 树的树根。

    \item 对于每层的环,$r$ 的祖先 $u$ 就是从 $r$ 开始沿树边走,第一个访问到的点,该环不要的边就是 $\text{in}(u)$。

    \item 从 $v=pre_u$ 开始依次访问环上的点,跳 $v=pre_v$,所有 $\text{in}(v)$ 就是环上的边,答案加上 $w(\text{in}(v))$。根据 $v$ 的入边可以得到进入环 $v$ 的点 $p$,递归处理 $p$(即展开环 $v$),注意 $p$ 只要跳到 $v$ 即可。

\end{enumerate}

展开过程 contraction 树上每条边只会访问一次,是 $\mathcal{O}(n)$ 的。

对于缩环过程,可以构造数据,使得 $\mathcal{O}(m)$ 条边需要从堆中 pop,时间复杂度 $\mathcal{O}(m \log m)$。

\begin{minted}{c++}
// == Preparations ==
#define pre(x) (id[e[in[x]].u])
typedef pair<int , int>pr;
int n , m , rt , num , h[N] , in[N] , f[N] , pre[N]; ll ans;
struct Edge{int u , v , w;}e[M + N];
struct Heap<pr>{}pq[N];//mergeable heap
struct UnionSet
{
    int fa[N];
    void init(int n){iota(fa + 1 , fa + n + 1 , 1);}
    int find(int x){return x == fa[x] ? x : fa[x] = find(fa[x]);}
    int& operator [](int x){return fa[find(x)];}
}col , id;
// == Main ==
bool Work(int u)
{
    while(pq[u] && (in[u] = pq[u].top().se , pre(u) == u))
        pq[u].pop();
    if(!pq[u])return 0;
    auto [w , i] = pq[u].top(); int v = pre(u);
    in[u] = i , pq[u].tag -= w , pq[u].pop();
    if(col[u] != col[v])
    {
        col[u] = col[v];
        return 1;
    }
    pq[++num].merge(pq[u]) , col[u] = col[num];
    for(int x = v ; x != u ; pre[x] = pre(x) , x = pre[x])
        f[x] = id[x] = num , pq[num].merge(pq[x]);
    f[u] = id[u] = num , pre[u] = v;
    return Work(num);
}
ll Expand(ll u , ll rt)
{
    ll ans = 0;
    for(; u != rt ; u = f[u])
    {
        for(int v = pre[u] ; v != u ; v = pre[v])
        {
            int w = e[in[v]].w;
            if(w >= INF)return -1;
            ans += Expand(e[in[v]].v , v) + w;
        }
    }
    return ans;
}
ll Solve(int n , int m , int rt , Edge e[N + M])//filled e[1...m]
{
    for(int i = 1 ; i <= n ; i++)e[++m] = {i % n + 1 , i , INF};
    for(int i = 1 ; i <= m ; i++)pq[e[i].v].push(pr(e[i].w , i));
    col.init(n * 2) , id.init(n * 2) , num = n;
    for(int i = 1 ; i <= n ; i++)if(id[i] == i)Work(i);
    return Expand(rt , num);
}
\end{minted}