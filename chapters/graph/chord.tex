\section{弦图}

\subsection{MCS 最大势算法。}

\begin{minted}{c++}
// == Preparations ==
#include <vector>

int pos[/* ... */], p[/* ... */];
vector<int> vec[/* ... */];
// == Main ==
for (int i = 1; i <= n; i++) pos[i] = vec[0].size(), vec[0].push_back(i);
for (int i = 1, j = 0; i <= n; i++, j++) {
    while (vec[j].empty()) j--;
    int u = p[i] = vec[j].back();
    vec[j].pop_back();
    pos[u] = -1;
    for (int k = g.hd[u]; k; k = g.nxt[k])
        if (pos[g.to[k]] != -1) {
            int v = g.to[k];
            pos[vec[l[v]].back()] = pos[v];
            swap(vec[l[v]][pos[v]], vec[l[v]].back());
            vec[l[v]].pop_back();
            pos[v] = vec[++l[v]].size();
            vec[l[v]].push_back(v);
        }
}
reverse(p + 1, p + n + 1);
\end{minted}

\subsection{弦图判定}

跑 MCS,然后判断是否为完美消除序列。

具体地,对于每个 $p _ i$,找到与之相连且在它之后出现的点,按出现顺序记为 $c _ 1, c _ 2, \ldots, c _ k$,我们只需要判断 $c _ 1$ 与 $c _ j$ 之间是否有边即可。因为这个团中其他边会在 $p _ {c _ 2}, p _ {c _ 3}, \ldots, p _ {c _ k}$ 中被判断。

\subsection{求弦图的团数与色数}

求团数:\\
设 $N(x)$ 为完美消除序列中在 $x$ 之后且与 $x$ 相连的点的集合,则弦图的最大团一定可以被表示为 $\{x\} + N(x)$,则 $|\{x\} + N(x)|$ 的最大值就是弦图的团数。

求色数:\\
考虑按完美消除序列从后往前考虑,贪心染 $\operatorname{mex}$,这样需要的颜色数量等于团数。  
由于团数小于等于色数,这样取到等号,一定最小。

\subsection{求弦图的最大独立集和最小团覆盖}

最大独立集:\\
按完美消除序列从前往后贪心。  
正确性证明:每次考虑最靠前的极大团,选最前面的点不劣于选其他点,且优于不选点。

最小团覆盖:\\
取最大独立集中的每个点 $x$ 对应的团 $\{x\} + N(x)$,这样需要的团的数量等于最大独立集的大小。  
由于最大独立集小于等于最小团覆盖,这样取到等号,一定最小。

\subsection{tricks}

区间图是弦图,完美消除序列为按区间右端点从小到大排序。

树上距离不超过 $k$ 的点连边是弦图,完美消除序列为 bfs 序的逆序。