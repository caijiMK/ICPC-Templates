\section{图计数相关}

\subsection{环计数}

任意环计数:状压 DP,设 $dp_{S, i}$ 为从 $S$ 中最小的点出发,走过的点集为 $S$,现在在 $i$ 的方案数,在能走到起点时统计答案,最终答案为减去边数再除 $2$。

三元环计数:将点按度数从小到大排序,然后边从前往后定向,这样每个点出度只有 $O(\sqrt{m})$ 条,枚举一个点 $u$,再枚举 $u$ 指向的点 $v$,再枚举 $v$ 指向的点 $w$,check 是否存在边 $(u, w)$ 即可。复杂度分析:当 $v$ 入度 $\le \sqrt{m}$ 时,$u$ 只有 $O(\sqrt{m})$ 个;当 $v$ 入度 $> \sqrt{m}$ 时,$w$ 至多 $O(\sqrt{m})$ 个。

竞赛图三元环计数:$\binom{n}{3} - \sum_{i = 1}^n \binom{d(i)}{2}$,其中 $d(i)$ 为 $i$ 的入度或出度。

四元环计数:同样的方法排序定向,然后枚举一个点 $a$,对所有 $c$ 记录 $a \to b \to c$ 的数量,然后对于每个 $c$ 任取两个 $b$ 即可组成一个四元环。

团计数:同样的方法排序定向,然后枚举一个点 $u$,对 $u$ 的出边 meet-in-middle。具体来说,对一个集合搜出每个子集是否为团,然后做高维前缀和,枚举另一个集合的子集,维护其是否为团及这个集合中的点与前一个集合中的点的连边的交即可。时间复杂度为 $O(\sqrt{m} \times 2^\frac{\sqrt{2m}}{2})$。(QOJ7514)

\subsection{Prufer 序列}

树到 Prufer 序列的映射:每次选择编号最小的叶子删掉并记录其父亲直到只剩两个点。

\begin{minted}{c++}
// == Main ==
// d[i] 为 i 的度数 -1
for (int i = 1, j = 1; i <= n - 2; i++, j++) {
    while (d[j]) j++;
    p[i] = fa[j];
    while (i <= n - 2 && !--d[p[i]] && p[i] < j) p[i + 1] = fa[p[i]], i++;
}
\end{minted}

Prufer 序列到树的映射:先算出度数,每次选择一个编号最小的度数为 $1$ 的点与当前 Prufer 序列的点连接,然后给两个点的度数都 $-1$,最后剩两个度数为 $1$ 的点连起来。

\begin{minted}{c++}
// == Main ==
// d[i] 为 i 的度数 -1
p[n - 1] = n;
for (int i = 1, j = 1; i < n; i++, j++) {
    while (d[j]) j++;
    fa[j] = p[i];
    while (i < n && !--d[p[i]] && p[i] < j) fa[p[i]] = p[i + 1], i++;
}
\end{minted}

给一张 $k$ 个连通块的图,每个连通块的点数为 $s_i$,求连 $k - 1$ 条边使其连通的方案数:

$$
\begin{aligned}
& \sum\limits_{d_i \ge 1, \sum_{i = 1}^k (d_i - 1) = k - 2} \binom{k - 2}{d_1 - 1, d_2 - 1, \cdots, d_k - 1} \cdot \prod\limits_{i = 1}^k s_i^{d_i} \\
=& \prod\limits_{i = 1}^k s_i \sum\limits_{d_i \ge 1, \sum_{i = 1}^k (d_i - 1) = k - 2} \binom{k - 2}{d_1 - 1, d_2 - 1, \cdots, d_k - 1} \cdot \prod\limits_{i = 1}^k s_i^{d_i - 1} \\
=& (\sum\limits_{i = 1}^k s_i)^{k - 2} \prod\limits_{i = 1}^k s_i \\
=& n^{k - 2} \prod\limits_{i = 1}^k s_i \\
\end{aligned}
$$

第二个等号为多元二项式定理。

\subsection{LGV 引理}

对于 DAG $G$:

\begin{itemize}
    \item 定义 $\omega{P}$ 为路径 $P$ 上所有边权之积。
    \item 定义 $e(u, v)$ 为从 $u$ 到 $v$ 所有路径的权值和,即 $e(u, v) = \sum_{P: u \to v} \omega(P)$。
    \item 起点集合为 $A$,终点集合为 $B$,满足 $|A| = |B| = n$。
    \item 定义一组 $A \to B$ 的不交路径 $S$:$S_i$ 是一条从 $A_i$ 到 $B_{\sigma(S)_i}$ 的路径($\sigma(S)$ 是一个排列),对于任意 $i \ne j$,有 $S_i$ 与 $S_j$ 没有公共顶点。
    \item 定义 $\mathrm{sign}(\sigma)$ 为排列 $\sigma$ 的逆序对数。
\end{itemize}

% 引理内容:

$$
\begin{aligned}
M = \left[
\begin{matrix}
e(A_1, B_1) & e(A_1, B_2) & \cdots & e(A_1, B_n) \\
e(A_2, B_1) & e(A_2, B_2) & \cdots & e(A_2, B_n) \\
\vdots & \vdots & \ddots & \vdots \\
e(A_n, B_1) & e(A_n, B_2) & \cdots & e(A_n, B_n) \\
\end{matrix}
\right] \\
\mathrm{det}(M) = \sum\limits_{S: A \to B} (-1)^{\mathrm{sign}(\sigma(S))} \prod\limits_{i = 1}^n \omega(S_i)
\end{aligned}
$$

\subsection{矩阵树定理}

允许重边,自环无影响。

\subsubsection{无向图}

令 $w(i, j)$ 为 $i, j$ 间的边的边权。

定义度数矩阵 $D(G)$ 为:

$$
D_{i, i}(G) = \sum\limits_{j} w(i, j), D_{i, j}(G) = 0, i \ne j
$$

定义邻接矩阵 $A(G)$ 为:

$$
A_{i, j}(G) = A_{j, i}(G) = w(i, j)
$$

定义 Laplace 矩阵 $L(G)$ 为 $L(G) = D(G) - A(G)$。

令一棵生成树的权值为边权之积,则这张图的所有生成树的权值和为 $L(G)$ 的任意主子式。

\subsubsection{有向图}

定义出度矩阵 $D^{\text{out}}(G)$ 为:

$$
D^{\text{out}}_{i, i}(G) = \sum\limits_{j} w(i, j), D^{\text{out}}_{i, j}(G) = 0, i \ne j
$$

类似的定义入度矩阵 $D^{\text{in}}(G)$。

定义邻接矩阵 $A(G)$ 为:

$$
A_{i, j}(G) = w(i, j)
$$

定义出度 Laplace 矩阵为 $L^{\text{out}}(G) = D^{\text{out}}(G) - A(G)$,类似地定义入度 Laplace 矩阵。

则以 k 为根的内向树的权值和为 $L^{\text{out}}(G)$ 的 $k$ 行 $k$ 列主子式,以 k 为根的外向树的权值和为 $L^{\text{in}}(G)$ 的 $k$ 行 $k$ 列主子式。

\subsection{BEST 定理}

对于有向欧拉图,其不同的欧拉回路总数为以任意点为根的内向/外向生成树个数乘上 $\prod_{v \in V} (deg(v) - 1)!$。