\section{网络流}

\subsection{上下界}

\begin{tcolorbox}
$f(u, v)$ 表示边 $(u, v)$ 的流量,$f(u)$ 表示 $u$ 的出流减入流,$c(u, v)$ 表示边 $(u, v)$ 的容量。

对于每条边给定一个流量下界 $b(u, v)$,需要额外满足 $\forall (u, v), b(u, v) \le f(u, v) \le c(u, v)$。
\end{tcolorbox}

\subsubsection{无源汇上下界可行流}

没有源点和汇点,对于所有点满足 $f(u) = 0$,求一个可行的流。

先强制每条边流到流量下界,建立虚拟源汇点 $s, t$,对于每个点 $u$ 考虑此时的净流量:

\begin{itemize}
    \item $f(u) = 0$:满足条件,不用管。
    \item $f(u) > 0$:出流大于入流,从 $u$ 向 $t$ 连容量为 $f(u)$ 的边。
    \item $f(u) < 0$:入流大于出流,从 $s$ 向 $u$ 连容量为 $-f(u)$ 的边。
\end{itemize}

将原图中每条边的容量设为 $c(u, v) - b(u, v)$,则从 $s$ 到 $t$ 的流相当于增加调整流量的过程。

若 $s$ 的出边流满(等同于 $t$ 的入边流满),则找到了一条可行流。

\subsubsection{有源汇上下界可行流}

连一条 $t$ 到 $s$ 容量正无穷下界为 $0$ 的边,然后跑无源汇上下界可行流即可,流量为新增边的流量。

\subsubsection{有源汇上下界最大流}

求出可行流后删掉 $t$ 到 $s$ 的边,在残量网络上跑 $s$ 到 $t$ 的最大流,该最大流加上原本的可行流即为答案。

\subsubsection{有源汇上下界最小流}

同理,改成求 $t$ 到 $s$ 的最大流,原可行流减去该最大流即为答案。

\subsubsection{有源汇上下界最小费用流}

做法是一样的,所有新增边费用为 $0$。

需要注意求最小流时需要改成费用最大。

\subsection{有负圈的最小费用最大流}

先钦定所有负圈边流满,即上下界均为流量。然后对于负边建反向、容量相同、费用为相反数的边用于退流原边。

这样就转化成了有源汇上下界最小费用最大流。