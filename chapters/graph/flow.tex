\section{网络流}

\subsection{最大流}

\begin{minted}{c++}
// == Preparations ==
#include <queue>
// == Main ==
struct Dinic {
    int s, t;
    struct graph {
        int tot, hd[205];
        int nxt[10005], to[10005], dt[10005];
        graph() {tot = 1;}
        void add(int u, int v, int w) {
            nxt[++tot] = hd[u];
            hd[u] = tot;
            to[tot] = v;
            dt[tot] = w;
            return ;
        }
    } g;
    int cur[205], dis[205];

    void add_edge(int u, int v, int f) {g.add(u, v, f), g.add(v, u, 0); return;}
    int bfs() {
        memset(dis, 0, sizeof(dis));
        queue<int>q;
        q.push(s);
        dis[s] = 1;
        while (!q.empty()) {
            int now = q.front();
            q.pop();
            cur[now] = g.hd[now];
            for (int i = g.hd[now]; i; i = g.nxt[i])
                if (g.dt[i] && !dis[g.to[i]]) dis[g.to[i]] = dis[now] + 1, q.push(g.to[i]);
        }
        return dis[t];
    }
    long long dinic(int now, long long flow) {
        if (now == t) return flow;
        long long used = 0;
        for (int i = cur[now]; i && used < flow; i = g.nxt[i])
            if (g.dt[i] && dis[g.to[i]] == dis[now] + 1) {
                long long k = dinic(g.to[i], min(flow - used, (long long)g.dt[i]));
                g.dt[i] -= k, g.dt[i ^ 1] += k;
                used += k;
                cur[now] = i;
            }
        if (used == 0) dis[now] = 0;
        return used;
    }
    long long solve() {
        long long ans = 0;
        while (bfs()) ans += dinic(s, 0x3f3f3f3f3f3f3f3f);
        return ans;
    }
} F;
\end{minted}

\subsection{费用流}

原始对偶:

\begin{minted}{c++}
// == Preparations ==
#include <queue>
// == Main ==
struct PrimalDual {
    int n, s, t;
    struct graph {
        int tot, hd[805];
        int nxt[30005], to[30005], flw[30005], cst[30005];

        graph() {tot = 1;}
        void add(int u, int v, int f, int c) {
            nxt[++tot] = hd[u];
            hd[u] = tot;
            to[tot] = v;
            flw[tot] = f;
            cst[tot] = c;
            return ;
        }
    } g;
    int h[805], dis[805], f[805], pre[805];
    struct node {
        int id, val;

        node() = default;
        node(int _id, int _val): id(_id), val(_val) {}
        bool operator<(const node &x) const {return val > x.val;}
    };

    void add_edge(int u, int v, int f, int c) {g.add(u, v, f, c), g.add(v, u, 0, -c); return;}
    void spfa() {
        queue<int> q;
        memset(h, 0x3f, sizeof(h));
        h[s] = 0;
        q.push(s);
        while (!q.empty()) {
            int now = q.front();
            q.pop();
            f[now] = 0;
            for (int i = g.hd[now]; i; i = g.nxt[i])
                if (g.flw[i] && h[g.to[i]] > h[now] + g.cst[i]) {
                    h[g.to[i]] = h[now] + g.cst[i];
                    if (!f[g.to[i]]) q.push(g.to[i]), f[g.to[i]] = 1;
                }
        }
        return ;
    }
    int dijkstra() {
        priority_queue<node> q;
        memset(dis, 0x3f, sizeof(dis));
        memset(pre, 0, sizeof(pre));
        q.emplace(s, dis[s] = 0);
        while (!q.empty()) {
            int now = q.top().id, tmp = q.top().val;
            q.pop();
            if (dis[now] != tmp) continue;
            for (int i = g.hd[now]; i; i = g.nxt[i])
                if (g.flw[i] && dis[g.to[i]] > dis[now] + g.cst[i] + h[now] - h[g.to[i]]) {
                    q.emplace(g.to[i], dis[g.to[i]] = dis[now] + g.cst[i] + h[now] - h[g.to[i]]);
                    pre[g.to[i]] = i ^ 1;
                }
        }
        return pre[t];
    }
    pair<int, int> solve() {
        int flow = 0, cost = 0;
        spfa();
        while (dijkstra()) {
            for (int i = 1; i <= n; i++)
                if (dis[i] < 0x3f3f3f3f) h[i] += dis[i];
            int mnflow = 0x3f3f3f3f;
            for (int i = t; i != s; i = g.to[pre[i]]) mnflow = min(mnflow, g.flw[pre[i] ^ 1]);
            for (int i = t; i != s; i = g.to[pre[i]]) g.flw[pre[i] ^ 1] -= mnflow, g.flw[pre[i]] += mnflow;
            flow += mnflow;
            cost += mnflow * h[t];
        }
        return {flow, cost};
    }
} F;
\end{minted}

\subsection{上下界}

\begin{tcolorbox}
$f(u, v)$ 表示边 $(u, v)$ 的流量,$f(u)$ 表示 $u$ 的出流减入流,$c(u, v)$ 表示边 $(u, v)$ 的容量。

对于每条边给定一个流量下界 $b(u, v)$,需要额外满足 $\forall (u, v), b(u, v) \le f(u, v) \le c(u, v)$。
\end{tcolorbox}

\subsubsection{无源汇上下界可行流}

没有源点和汇点,对于所有点满足 $f(u) = 0$,求一个可行的流。

先强制每条边流到流量下界,建立虚拟源汇点 $s, t$,对于每个点 $u$ 考虑此时的净流量:

\begin{itemize}
    \item $f(u) = 0$:满足条件,不用管。
    \item $f(u) > 0$:出流大于入流,从 $u$ 向 $t$ 连容量为 $f(u)$ 的边。
    \item $f(u) < 0$:入流大于出流,从 $s$ 向 $u$ 连容量为 $-f(u)$ 的边。
\end{itemize}

将原图中每条边的容量设为 $c(u, v) - b(u, v)$,则从 $s$ 到 $t$ 的流相当于增加调整流量的过程。

若 $s$ 的出边流满(等同于 $t$ 的入边流满),则找到了一条可行流。

\subsubsection{有源汇上下界可行流}

连一条 $t$ 到 $s$ 容量正无穷下界为 $0$ 的边,然后跑无源汇上下界可行流即可,流量为新增边的流量。

\subsubsection{有源汇上下界最大流}

求出可行流后删掉 $t$ 到 $s$ 的边,在残量网络上跑 $s$ 到 $t$ 的最大流,该最大流加上原本的可行流即为答案。

\subsubsection{有源汇上下界最小流}

同理,改成求 $t$ 到 $s$ 的最大流,原可行流减去该最大流即为答案。

\subsubsection{有源汇上下界最小费用流}

做法是一样的,所有新增边费用为 $0$。

需要注意求最小流时需要改成费用最大。

\subsection{有负圈的最小费用最大流}

先钦定所有负圈边流满,即上下界均为流量。然后对于负边建反向、容量相同、费用为相反数的边用于退流原边。

这样就转化成了有源汇上下界最小费用最大流。