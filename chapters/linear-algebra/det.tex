\section{行列式}

\begin{tcolorbox}
    \textbf{行列式}:一个 $n$ 阶方阵 $A$ 的行列式定义为
    $$\sum_{\sigma\in S_n} (-1)^{\operatorname{N}(\sigma)}\prod_{i=1}^{n} A_{i,\sigma_i}$$
    其中 $S_n$ 是所有 $n$ 阶排列的全体,$\operatorname{N}(\sigma)$ 表示 $\sigma$ 的逆序对数。

    我们用 $\det(A)$ 或 $|A|$ 表示 $A$ 的行列式。
\end{tcolorbox}

\subsection{行列式的基本性质}

\begin{itemize}
    \item 某一行(列)有公因子 $k$,则可以提出 $k$。
    \item 两行(列)互换,改变行列式正负符号:有向面积的方向改变。
    \item 在行列式中,有两行(列)对应成比例或相同,则此行列式的值为 $0$:有两个重叠的向量,有向面积为零。
    \item 在行列式中,某一行(列)的每个元素是两数之和,则此行列式可拆分为两个相加的行列式:一行只能选一个数,这个数拆成和的左 / 右部分。
    $$
            \begin{aligned}
                \begin{vmatrix}
                    a_{1,1} & a_{1,2} & \cdots & a_{1,n} \\
                    \vdots & \vdots & \ddots & \vdots \\
                    a_{i,1}+b_{i,1} & a_{i,2}+b_{i,2} & \cdots & a_{i,n}+b_{i,n} \\
                    \vdots & \vdots & \ddots & \vdots \\
                    a_{n,1} & a_{n,2} & \cdots & a_{n,n}
                \end{vmatrix} 
                =
                \begin{vmatrix}
                    a_{1,1} & a_{1,2} & \cdots & a_{1,n} \\
                    \vdots & \vdots & \ddots & \vdots \\
                    a_{i,1} & a_{i,2} & \cdots & a_{i,n} \\
                    \vdots & \vdots & \ddots & \vdots \\
                    a_{n,1} & a_{n,2} & \cdots & a_{n,n}
                \end{vmatrix}
                +
                \begin{vmatrix}
                    a_{1,1} & a_{1,2} & \cdots & a_{1,n} \\
                    \vdots & \vdots & \ddots & \vdots \\
                    b_{i,1} & b_{i,2} & \cdots & b_{i,n} \\
                    \vdots & \vdots & \ddots & \vdots \\
                    a_{n,1} & a_{n,2} & \cdots & a_{n,n}
                \end{vmatrix}
            \end{aligned}
    $$
    \item 将一行(列)的 $k$ 倍加进另一行(列)里,行列式的值不变:可用上两条性质证明。
    \item 将行列式的行列互换,行列式的值不变。
\end{itemize}

\subsection{余子式与拉普拉斯展开}

\begin{tcolorbox}
    对于方阵 $A$,$I=\{i_1,i_2,\ldots,i_k\}\subseteq R$,$J=\{j_1,j_2,\ldots,j_k\}\subseteq C$,定义:

    \begin{itemize}
        \item $|A[I,J]|$ 称为子式,即选出一个子方阵求行列式。若 $I=J$(即选出的行、列集合相同),则也可称为主子式。当 $I=J=\{1,2,\ldots,k\}$ 时,也称为 $k$ 阶顺序主子式。
        \item $|A[R\setminus I,C\setminus J]|$ 称为余子式。
        \item $\operatorname{sgn}(I,J)|A[R\setminus I,C\setminus J]|$ 称为余子式。
    \end{itemize}
\end{tcolorbox}

其中 $\operatorname{sgn}(I,J)=(-1)^{i_1+\cdots+i_k+j_1+\cdots+j_k}$,下同。

矩阵的秩等于最高阶非零子式的阶数。这是因为如果子矩阵存在线性相关的两行(列),则行列式为零。

\begin{tcolorbox}
    \textbf{拉普拉斯展开}:对于 $n$ 阶方阵 $A$,对于任意 $1\le i\le n$,有
    $$|A|=\sum_{j=1}^{n} (-1)^{i+j}|A[R\setminus\{i\},C\setminus \{j\}]|A_{i,j}$$
\end{tcolorbox}
\begin{itemize}
    \item 对列也有同样的性质。
    \item 根据行列式的定义即可证明。
\end{itemize}

\begin{tcolorbox}
    \textbf{拉普拉斯定理}:对于方阵 $A$,以及任意 $I\subseteq R$,有
    $$|A|=\sum_{J\subseteq C,|J|=|I|} \operatorname{sgn}(I,J)|A[I,J]|\cdot|A[R\setminus I,C\setminus J]|$$
\end{tcolorbox}

\begin{minted}{c++}
// == Preparations ==
typedef long long ll;
const int N = 605;
// == Main ==
int Det(int n , int mod , int a[N][N])
{
    int m = 0 , f = 1;
    auto Swap = [&](int i , int j)
    {
        if(i == j)return ;
        for(int k = 1 ; k <= n ; k++)
            swap(a[i][k] , a[j][k]);
        f = -f;
    };
    for(int i = 1 ; i <= n ; i++)
    {
        m++;
        for(int j = m ; j <= n ; j++)
            if(a[j][i]){Swap(m , j); break ;};
        if(!a[m][i])return 0;
        for(int j = m + 1 ; j <= n ; j++)
        {
            while(a[j][i] && a[m][i])
            {
                ll r = a[j][i] / a[m][i] % mod;
                for(int k = 1 ; k <= n ; k++)
                    a[j][k] = (a[j][k] - r * a[m][k]) % mod;
                Swap(m , j);
            }
            if(!a[m][i])Swap(m , j);
        }
    }
    ll ans = 1;
    for(int i = 1 ; i <= n ; i++)
        (ans *= a[i][i]) %= mod;
    return (ans * f % mod + mod) % mod;
}
\end{minted}