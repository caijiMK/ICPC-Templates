\section{相似矩阵}

\begin{tcolorbox}
\textbf{矩阵的相似}:设 $A,B$ 是定义在数域 $K$ 上的 $n$ 阶方阵,若存在 $K$ 上的可逆矩阵 $U$ 满足 $B=U^{-1}AU$ 
            则称 $A$ 与 $B$ 相似,记作 $A\sim B$。
\end{tcolorbox}

$\mathcal{M}_n(K)$ 上的相似关系是等价关系。

对于矩阵 $A,B$,若 $A\sim B$,则有:

\begin{itemize}
    \item $\operatorname{rank}(A)=\operatorname{rank}(B)$
    \item $\det(A)=\det(B)$
    \item $\operatorname{tr}(A)=\operatorname{tr}(B)$
\end{itemize}

\begin{tcolorbox}
\textbf{对角化}:矩阵 $A$ 与对角矩阵 $D$ 相似,即 $A=UDU^{-1}$,等价于 $AU=UD$ 且 $U$ 可逆:
\end{tcolorbox}

\section{特征值与特征多项式}

\begin{tcolorbox}
\textbf{特征值}:设 $A$ 是数域 $K$ 上的 $n$ 阶方阵。若有向量 $\alpha \in K^n,\alpha\ne 0$ 和数 $\lambda\in K$,使得 $A\alpha=\lambda\alpha$ 则称 $\lambda$ 是 $A$ 的特征值,$\alpha$ 是 $A$ 的特征向量,或者说 $\alpha$ 是属于 $\lambda$ 的特征向量。
\end{tcolorbox}

\begin{tcolorbox}
\textbf{定理}:$\lambda$ 是 $A$ 的特征值当且仅当 $|\lambda I-A|=0$,$\alpha$ 是属于 $\lambda$ 的特征向量当且仅当 $\alpha$ 是 $(\lambda I-A)x=0$ 的非零解。
\end{tcolorbox}

\begin{tcolorbox}
\textbf{特征多项式}:$p_A(x)=|xI-A|$。其为一个 $n$ 次多项式,在复数域上恰有 $n$ 个根(包括重根)。
\end{tcolorbox}

由特征多项式的定义可得 $[x^n]p_A(x)=1,[x^{n-1}]p_A(x)=-\operatorname{tr}(A),[x^0]p_A(x)=(-1)^n\det(A)$。

\subsection{求解特征多项式}

由于相似的两个矩阵拥有相同的特征多项式,所以我们可以求出原矩阵 $A$ 的一个相似矩阵 $B$,如果 $p_B(x)$ 可以快速计算,那么就可以得到 $p_A(x)$。

考虑初等变换的影响 $PAP^{-1}$,最终可以得到一个上海森堡矩阵,其求行列式是 $\mathcal{O}(n^2)$ 的,所以可以在 $\mathcal{O}(n^3)$ 复杂度内求解一个矩阵的特征多项式。

\begin{minted}{c++}
// == Preparations ==
int Inv(int x);
// == Main ==
int Hessenberg(int a[][N] , int n)
{
    static int f[N]; f[n + 1] = 1;
    for(int i = n ; i >= 1 ; i--)
    {
        int aa = 1; f[i] = 0;
        for(int j = i ; j <= n ; j++)
        {
            f[i] = (f[i] + (ll)((j - i) & 1 ? -1 : 1) * a[i][j] * aa % MOD * f[j + 1]) % MOD;
            aa = (ll)aa * a[j + 1][j] % MOD;
        }
    }
    return f[1];
}
Poly operator * (const Poly &a , const Poly &b)
{
    int n = siz(a) , m = siz(b);
    Poly c(n + m - 1);
    for(int i = 0 ; i < n ; i++)
        for(int j = 0 ; j < m ; j++)
            c[i + j] = (c[i + j] + (ll)a[i] * b[j]) % MOD;
    return c;
}
Poly operator * (const Poly &a , const int &v)
{
    Poly c = a;
    for(int i = 0 ; i < siz(a) ; i++)
        c[i] = (ll)c[i] * v % MOD;
    return c;
}
Poly operator + (const Poly &a , const Poly &b)
{
    int n = siz(a) , m = siz(b);
    Poly c = a; c.resize(max(n , m));
    for(int i = 0 ; i < m ; i++)c[i] = (c[i] + b[i]) % MOD;
    return c;
}
Poly Lagrange(int y[N] , int n)
{
    static int nn , fac[N] , inv[N] , t[N]; 
    static Poly pre[N] , g[N];
    if(nn < n)
    {
        fac[0] = inv[0] = 1;
        for(int i = 1 ; i <= n ; i++)fac[i] = (ll)fac[i - 1] * i % MOD;
        inv[n] = Inv(fac[n]);
        for(int i = n - 1 ; i >= 1 ; i--)inv[i] = (ll)inv[i + 1] * (i + 1) % MOD;
        nn = n;
    }
    Poly res; pre[0] = {1};
    for(int i = 1 ; i <= n ; i++)g[i] = {-i , 1};
    for(int i = 1 ; i <= n ; i++)pre[i] = pre[i - 1] * g[i];
    for(int i = 1 ; i <= n ; i++)t[i] = (__int128)y[i] * inv[i - 1] * inv[n - i] * ((n - i) & 1 ? -1 : 1) % MOD;
    for(int i = 1 ; i <= n ; i++)res = pre[i - 1] * t[i] + res * g[i];
    return res;
}// poly need to mod
Poly Characteristic(int a[][N] , int n)
{
    static int y[N] , h[N][N];
    for(int i = 1 ; i < n ; i++)
    {
        for(int j = i + 1 ; j <= n ; j++)
        {
            if(a[j][i])
            {
                for(int k = 1 ; k <= n ; k++)
                    swap(a[i + 1][k] , a[j][k]);
                for(int k = 1 ; k <= n ; k++)
                    swap(a[k][i + 1] , a[k][j]);
                break ;
            }
        }
        if(!a[i + 1][i])continue ;
        for(int j = i + 2 ; j <= n ; j++)
        {
            int v = (ll)a[j][i] * Inv(a[i + 1][i]) % MOD;
            for(int k = 1 ; k <= n ; k++)
                a[j][k] = (a[j][k] - (ll)a[i + 1][k] * v) % MOD;
            for(int k = 1 ; k <= n ; k++)
                a[k][i + 1] = (a[k][i + 1] + (ll)a[k][j] * v) % MOD;
        }
    }
    for(int i = 1 ; i <= n ; i++)
        for(int j = 1 ; j <= n ; j++)
            h[i][j] = -a[i][j];
    for(int x = 1 ; x <= n + 1 ; x++)
    {
        for(int i = 1 ; i <= n ; i++)h[i][i]++;
        y[x] = Hessenberg(h , n);
    }
    return Lagrange(y , n + 1);
}
\end{minted}