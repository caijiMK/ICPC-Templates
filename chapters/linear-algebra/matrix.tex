\section{矩阵}

\subsection{矩阵乘法的另一种理解}

\begin{itemize}
    \item $A$ 右乘矩阵 $B$,是对 $A$ 的列向量进行线性组合,组合系数记在 $B$ 的每个列向量中。
    \item $B$ 左乘矩阵 $A$,是对 $B$ 的行向量进行线性组合,组合系数记在 $A$ 的每个行向量中。
\end{itemize}

\subsection{矩阵的秩}

\begin{tcolorbox}
    \textbf{矩阵的行秩}:一个矩阵 $A$ 的行秩为 $A$ 的行向量的极大线性无关组的大小。

    \textbf{矩阵的列秩}:列秩是 $A$ 的列向量的极大线性无关组的大小。
\end{tcolorbox}

\begin{tcolorbox}
    \textbf{定理}:矩阵的行秩等于列秩。
\end{tcolorbox}

\begin{tcolorbox}
    \textbf{矩阵的秩}:将矩阵 $A$ 的行秩和列秩统称为矩阵 $A$ 的秩,记为 $\operatorname{rank}(A)$。
\end{tcolorbox}

\begin{tcolorbox}
    \textbf{定理}:对于任意矩阵 $A,B$,有 $\operatorname{rank}(AB)\le \min(\operatorname{rank}(A),\operatorname{rank}(B))$.
    \textbf{定理}:对于一个 $n\times m$ 的矩阵 $A$,假设其秩为 $r$,则 $A$ 可以表示为一个 $n\times r$ 的矩阵和一个 $r\times m$ 的矩阵的乘积。
\end{tcolorbox}


\subsection{基本子空间}

\begin{tcolorbox}
    \textbf{行空间}:矩阵 $A$ 的行向量组的生成子空间称为行空间,记为 $C(A^{\mathrm{T}})$。

    \textbf{列空间}:矩阵 $A$ 的列向量组的生成子空间称为列空间,记为 $C(A)$。
\end{tcolorbox}
\begin{tcolorbox}
    根据矩阵的秩的定义,有:对于任意矩阵 $A$,有 $\dim C(A^{\mathrm{T}})=\dim C(A)=\operatorname{rank}(A)$。
\end{tcolorbox}

\begin{tcolorbox}
    \textbf{零空间}:矩阵 $A$ 的零空间为 $Ax=0$ 的所有解构成的空间,记为 $N(A)$。

    \textbf{左零空间}:矩阵 $A$ 的左零空间为 $A^{\mathrm{T}}x=0$(或表述为 $x^{\mathrm{T}}A=0$) 的所有解构成的空间,记为 $N(A^{\mathrm{T}})$。
\end{tcolorbox}

\begin{tcolorbox}
    有:对于任意 $n\times m$ 的矩阵 $A$,有 $\dim N(A)=m-\operatorname{rank}(A),\dim N(A^{\mathrm{T}})=n-\operatorname{rank}(A)$。
\end{tcolorbox}

\begin{tcolorbox}
    \textbf{四个基本子空间的正交性}:根据四个子空间的定义,对于任意矩阵 $A$ 有:
    \begin{itemize}
        \item $N(A)$ 与 $C(A^{\mathrm{T}})$ 互为正交子空间,且互为正交补。
        \item $N(A^{\mathrm{T}})$ 与 $C(A)$ 互为正交子空间,且互为正交补。
    \end{itemize}

\end{tcolorbox}

\subsection{高斯消元}

\begin{minted}{c++}
// == Preparations ==
double a[M][N];
int Cmp(double x , double y = 0){return abs(x - y) < 1e-8 ? 0 : (x < y ? -1 : 1);}
// == Main ==
int Gauss()
{
    int tot = 0;
    for(int i = 1 ; i <= n ; i++)
    {
        for(int j = tot + 1 ; j <= m ; j++)
        {
            if(Cmp(a[j][i]))
            {
                for(int k = 1 ; k <= n + 1 ; k++)
                    swap(a[tot + 1][k] , a[j][k]);
                break ;
            }
        }
        if(!Cmp(a[tot + 1][i]))continue ;
        for(int j = tot + 2 ; j <= m ; j++)
        {
            double r = a[j][i] / a[tot + 1][i];
            for(int k = 1 ; k <= n + 1 ; k++)
                a[j][k] -= r * a[tot + 1][k];   
        }
        tot++;
    }
    for(int i = tot + 1 ; i <= m ; i++)
    {
        if(Cmp(a[i][n + 1]))
        {
            return -1;// No solutions
        }
    }
    if(tot == n)
    {
        for(int i = n ; i >= 1 ; i--)
        {
            a[i][n + 1] /= a[i][i] , a[i][i] = 1;   
            for(int j = i - 1 ; j >= 1 ; j--)
                a[j][n + 1] -= a[j][i] * a[i][n + 1] , a[j][i] = 0;
        }
        return 0;// One solution
    }
    return 1;// Many solutions
}
\end{minted}

\subsection{逆矩阵}

\begin{tcolorbox}
    \textbf{逆矩阵}:设 $A$ 是一个 $n$ 阶方阵,若存在 $n$ 阶方阵 $B$ 使得 $AB=BA=I_n$,其中 $I_n$ 为 $n$ 阶单位矩阵,则称 $A$ 是可逆的,是非奇异矩阵,$B$ 是 $A$ 的逆矩阵,记为 $A^{-1}$。否则称 $A$ 不可逆,是奇异矩阵。
\end{tcolorbox}

逆矩阵可以用来快速求解线性方程组。设有 $Ax=b$,则有 $x=A^{-1}b$,对于不同的 $b$,每次只需要 $\mathcal{O}(n^2)$ 即可计算一组解。这样做的前提是 $A$ 可逆。

\begin{minted}{c++}
// == Preparations ==
const int MOD = 1e9 + 7;
int Inv(int x);
// == Main ==
bool Inv(int n , int a[N][N * 2])
{
    for(int i = 1 ; i <= n ; i++)
        a[i][i + n] = 1;
    for(int i = 1 ; i <= n ; i++)
    {
        for(int j = i ; j <= n ; j++)
        {
            if(a[j][i] != 0)
            {
                for(int k = 1 ; k <= n * 2 ; k++)
                    swap(a[j][k] , a[i][k]);
                break ;
            }
        }
        if(!a[i][i])return 0;// No Solution
        int inv = Inv(a[i][i]);
        for(int j = 1 ; j <= n ; j++)
        {
            if(j == i)continue ;
            int r = (ll)a[j][i] * inv % MOD;
            for(int k = 1 ; k <= n * 2 ; k++)
                a[j][k] = (a[j][k] - (ll)a[i][k] * r) % MOD;
        }
        for(int j = 1 ; j <= n * 2 ; j++)
            a[i][j] = (ll)a[i][j] * inv % MOD;
    }
    for(int i = 1 ; i <= n ; i++)
        for(int j = 1 ; j <= n ; j++)
            if(a[i][j + n] < 0)a[i][j + n] += MOD;
    return 1;
}
\end{minted}